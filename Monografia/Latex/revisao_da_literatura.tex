\chapter[Revisão de Literatura]{Revisão de Literatura}

%\addcontentsline{toc}{chapter}{Revisão de Literatura}
% ----------------------------------------------------------

Nesse capítulo é feita uma revisão no estado da arte dos algoritmos de roteamento de veículos, e a aplicação de algoritmos genéticos para mesma finalidade.

\section{Roteamento de Veículos com Janelas de Tempo}

Um dos problemas mais importantes de otimização combinatória e mais estudados na literatura de pesquisa operacional é o problema de Roteamento de Veículos com Janelas de Tempo (PRVJT).
Nele consiste que tendo uma frota de veículos que deve partir de um depósito, deve atender a demanda de N consumidores e retornar ao depósito de forma que o custo total de viagem seja o mínimo. Levando em consideração que o atendimento aconteça dentro de um intervalo de tempo especificado para cada consumidor. Também deve-se respeitar a capacidade dos veículos.

Na literatura existem vários objetivos abordados pelos autores para o PRVJT.\ Neste trabalho temos como objetivo a minimização da distancia total percorrida, que é o mais comum na literatura.~\cite{ROCHAT}

\subsection{Formulação matemática}

O PRVJT pode ser definido a partir um grafo completo orientado~\(G = (V,A)\) em que \(V = {0,\cdots,n+1}\) é um conjunto de vértices e \(G ={(i,j)|i,j \in V}\) é o conjunto de arcos.
Cada arco (i,j) é associado a um tempo \(t_{ij}\) e um custo de travessia \(c_{ij}\).

% https://en.wikibooks.org/wiki/LaTeX/Mathematics

\section{Algoritmos genéticos}

AG é uma técnica amplamente utilizada de IA, que utilizam conceitos provenientes do princípio de seleção natural para abordar uma  ampla série de problemas, geralmente de adaptação. \cite{DiogoCLucas}

\subsection{Funcionamento}
 
Inspirado na maneira como o seleção natural explica o processo de evolução das espécies, Holland \cite{Holland1975} decompôs o funcionamento dos AG em sete etapas, essa são \textit{inicialização}, \textit{avaliação}, \textit{seleção}, \textit{cruzamento}, \textit{mutação}, \textit{atualização} e  \textit{finalização} conforme a Figura \ref{fig:EstruturaAG}. 

\begin{minipage}{\linewidth}
	\makebox[\linewidth]{
		\includegraphics[keepaspectratio=true,scale=0.45]{ibagens/genetico1.png}}
	\captionof{figure}{Estrutura de um AG \cite{DiogoCLucas} }
	\label{fig:EstruturaAG}
\end{minipage}


\subsection{Inicialização}
Criar uma população de possíveis respostas para um problema. 
É comum fazer uso de funções aleatórias para gerar os indivíduos, sendo este um recurso simples que visa fornecer maior diversidade.

\subsection{Avaliação}
Avalia-se a aptidão das soluções, os indivíduos da população, então é feita uma análise para que se estabeleça quão bem elas respondem ao problema proposto.
A função de avaliação também pode ser chamada de função objetivo. Ela pode variar de acordo com problema,  
Calcular com exatidão completa o grau de adaptação dos indivíduos pode ser uma tarefa complexa em muitos casos, e se levarmos em conta que esta operação é repetida varias vezes ao longo do processo de evolução, seu custo pode ser consideravelmente alto. Em tais situações é comum o uso de funções não determinísticas, que não avaliam a totalidade das características do indivíduo, operando apenas sobre uma amostragem destas.

\subsection{Seleção}
Ela é a responsável pela perpetuação de boas características na espécie. 
Neste estágio que os indivíduos são escolhidos para posterior cruzamento, fazendo uso do grau de adaptação de cada um é realizado um sorteio, onde os indivíduos com maior grau de adaptação tem maior probabilidade de se reproduzirem.
O grau adaptação é calculado a partir da função de avaliação para cada individuo, determina o quão apto ele esta para reprodução relativo a sua população. 

\textbf{Selection Random}: Gera um numero aleatório entre 0 e o tamanho total da população e retorna o indivíduo do índice escolhido.

\textbf{Selection Roulette Wheel}: Faz a soma de todos os valores da função de aptidão da população, depois calcula a porcentagem de cada indivíduo referente ao total 
e guarda em um vetor. Então é gerado um valor A aleatório entre 0 e 1 e multiplicado pelo valor total dos pesos. Para selecionar o indivíduo é feito um loop 
nos pesos e seus valores somados até que o valor A seja igual ou menor que zero, o índice do peso que fez a condição acontecer, se o índice do indivíduo selecionado.
Desta forma aumentando a possibilidade de selecionar um indivíduo com maior aptidão.

\subsection{Cruzamento}
Características das soluções escolhidas na seleção são recombinadas, gerando novos indivíduos.

\textbf{CrossOver Simple}: Utiliza dois indivíduos selecionados,  define dois números aleatórios de 0 até menor tamanho da lista de cromossomos entre os dois, 
sendo que o primeiro índice tem que ser menor que o segundo índice e os mesmos não podem ser iguais. 
Esse tamanho é utilizado para trocar cromossomos entre os dois indivíduos, ou seja, adicionar todos os cromossomos do primeiro indivíduo do índice 
igual ao primeiro numero, até o índice segundo numeSAro, e repete o processo contrario.

\textbf{Crossover OBX (Order-Based Crossover)}: Utiliza dois indivíduos escolhidos na seleção, então define dois números aleatórios, de 0 até menor tamanho da lista de cromossomos entre os dois, sendo que o primeiro tem que ser menor que o segundo e não podem ser iguais. O primeiro numero até o segundo numero, são definidas posições aleatórias e são salvas em uma lista. Faz um loop na lista e troca o cromossomo da posição do primeiro indivíduo para o segundo e do segundo para o primeiro.

\textbf{Crossover PBX (Position-Based Crossover)}: Utiliza dois indivíduos selecionados, então define dois números aleatórios, 
de 0 até menor tamanho da lista de cromossomos entre os dois, sendo que o primeiro índice tem que ser menor que o segundo índice e os mesmos não podem ser iguais. 
Entre esse tamanho são definidas posições aleatórias e guardadas em uma lista. Os indivíduos resultantes são zerados, 
e para cada posição é trocado do cromossomo principal para o resultante de mesma posição outro da mesma posição. 
As posições não preenchidas são completadas com os cromossomos restante, seguindo a ordem do cromossomo e adicionado se ele não ja existir na lista.

\subsection{Mutação}
Características dos indivíduos resultantes do processo de reprodução são alteradas, acrescentando assim variedade a população.
A mutação opera sobre os indivíduos resultantes do processo de cruzamento e com uma probabilidade pré-determinada efetua algum tipo de alteração em sua  estrutura. A importância desta operação é o fato de que uma vez bem escolhido seu modo de atuar, é garantido que diversas alternativas serão exploradas.

\textbf{MutateEM (Exchange Mutation)}: Define duas das posições aleatórias distintas do segundo cromossomo até o ultimo, e troca os cromossomos do indivíduo.

\textbf{MutateSM (Scramble Mutation)}: Define duas das posições aleatórias distintas do segundo cromossomo até o ultimo, e uma quantidade aleatória. Então faz um loop da quantidade aleatória e mistura os cromossomos que estão entre a posição inicial e final trocando aleatoriamente dois pontos entre eles.

\textbf{MutateDM (Displacement Mutation)}: Define duas das posições aleatórias distintas do segundo cromossomo até o ultimo, e remove todos os cromossomo entre essa posições e recoloca a partir de uma posição aleatória.

\textbf{MutateIM (Insertion Mutation)}: Define uma posição aleatória, remove o cromossomo da posição, reorganiza os cromossomos e insere o cromossomo removido em uma nova posição aleatória.

\textbf{MutateIVM (Inversion Mutation)}: Define duas das posições aleatórias distintas do segundo cromossomo até o ultimo, e inverte todos os cromossomos que está entre as posições.

\textbf{MutateDIVM (Displaced Inversion Mutation)}:Define duas das posições aleatórias distintas do segundo cromossomo até o ultimo, e remove todos os cromossomo entre essa posições e recoloca a partir de uma posição aleatória de forma invertida.

\subsection{Atualização}
Os indivíduos criados no processo de reprodução e mutação são inseridos na população.

Na forma mais tradicional deste a população mantém um tamanho fixo e os indivíduos são criados em mesmo número que seus antecessores e os substituem por completo. 

Existem, porém, algumas alternativas, o número de indivíduos gerados pode ser menor ou o tamanho da população pode sofrer variações e o critério de inserção pode variar, por exemplo, nos casos em que os filhos substituem os pais, ou em que estes só são inseridos se possuírem maior aptidão que o cromossomo que sera substituído, ou o manter sempre o conjunto dos n melhores indivíduos. 

\subsection{Finalização}
É testado se as condições de encerramento da evolução foram atingidas, retornando para a etapa de avaliação em caso negativo e encerrando a execução em caso positivo.

Os critérios para a parada podem ser vários, desde o número de gerações criadas até o grau de convergência da população atual.


Toda base dos AG se fundamenta nos indivíduos, eles são a unidade básica em qual o algoritmo se baseia, sua função é codificar as possíveis soluções do problema a ser tratado e partir de sua manipulação no processo evolutivo, a partir daí que são encontradas as respostas.

Esses indivíduos precisam de uma representação, essa será o principal responsável pelo desempenho do programa. É comum chamar de \textit{genoma} ou \textit{cromossomo} para se referir ao individuo. Por essa definição podemos resumir um indivíduo pelos genes que possui, ou seja seu \textit{genótipo}.

Apesar de toda representação por parte do algoritmo ser baseada única e exclusivamente em seu genótipo, toda avaliação é baseada em seu fenótipo, o conjunto de características observáveis no objeto resultante do processo de decodificação dos genes do individuo, ver Tabela 1.

\newcolumntype{C}[1]{>{\centering\let\newline\\\arraybackslash\hspace{0pt}}m{#1}}
\begin{table}[h]
	\centering
\vspace{0.5cm}
\renewcommand{\arraystretch}{2.0}
\caption{Exemplos de genótipos e fenótipos correspondentes em alguns tipos de problemas \cite{DiogoCLucas}}
	\begin{tabular}{|C{4cm}|C{3.5cm}|C{7cm}|}
		\hline
		\textbf{Problema} & \textbf{Genótipo} & \textbf{ Fenótipo} \\
		\hline                  
		Otimização numérica & 0010101001110101 & 10869 \\
		\hline
		Caixeiro viajante & CGDEHABF & Comece pela cidade C, depois passe pelas cidades G, D, E, H, A, B e termine em F \\
		\hline
		Regras de aprendizado para agentes & C$_1$R$_4$C$_2$R$_6$C$_4$R$_1$ & Se condição 1 (C$_1$) execute regra 4 (R$_4$), se (C$_2$) execute (R$_6$), se (C$_4$) execute (R$_1$)\\
		\hline
	\end{tabular}
\end{table}	

Para cada indivíduo é calculado o seu grau de adaptação, a partir de uma função objetivo, comumente denotada como na formula \ref{eq:solve0}.

\begin{equation} \label{eq:solve0}
f_O(x)  
\end{equation}


Que vai representar o quão bem a resposta apresentada pelo individuo soluciona o problema proposto.

Também é calculado o grau de adaptação do indivíduo relativo aos outros membros da população a qual ele pertence, esse é chamado de grau de aptidão, para um indivíduo $x$ temos seu grau de aptidão denotado pela fórmula \ref{eq:solve1}.


\begin{equation} \label{eq:solve1}
	f_A(x) = \frac{f_O(x)}{ \sum_{i=1}^{n}  f_O(i)  }  
\end{equation}


 Sendo n o tamanho da população.
 
 A dinâmica populacional é a responsável pela evolução, ao propagar características desejáveis a gerações subsequentes no processo de cruzamento, enquanto novas são testadas no processo de mutação.
 
 Algumas definições importantes relativo as populações de um AG são:
 
 \textbf{Geração:} É o número de vezes em que a população passou pelo processo de seleção, reprodução, mutação e atualização.

\textbf{Média de adaptação:} É a taxa média que ao indivíduos se adaptaram ao problema, é definida pela formula \ref{eq:solve2}. 

\begin{equation} \label{eq:solve2}
M_A = \frac{ \sum_{i=1}^{n} f_O(i) }{n}
\end{equation}


\textbf{Grau de convergência:} define o qual próxima esta a media de adaptação desta população relativo as anteriores. O objetivo dos AG é fazer a população convergir para uma valor de adaptação ótimo.
Um estado negativo que pode ocorrer relativo a esta medida é a \textit{convergência prematura}, a mesma ocorre quando a população converge em uma média de adaptação sub-ótima, e dela não consegue sair por causa de sua baixa diversidade.

\textbf{Diversidade:} Mede o grau de variação entre os genótipos da população. Ela é fundamental para o tamanho da busca.
Sua queda esta fortemente ligada ao fenômeno de \textit{Convergência prematura}.

\textbf{Elite:} São os indivíduos mais bem adaptados da população. Uma técnica comum nos AG é p \textit{elitismo}, onde são selecionados k melhores indivíduos que serão mantidos a cada geração.

\subsection{Aplicações}
Existem vários aplicações para os algoritmo genéticos, por serem uma inteligência artificial não supervisionada, de rápido aprendizado e podendo ser paralelizado.

O modelo m-PRC(Problema de Rotas de Cobertura multi-veículo) é uma aplicação de algoritmos genéticos para construção de rotas em uma região mapeada, para encontrar uma boa distribuição de viaturas para patrulhamento urbano usado por departamentos de segurando como a policia, guardas municipais ou segurança privada \cite{Washington}. 
O Modelo é definido como um grafo não direcionado \ref{eq:solve3}. 

\begin{equation} \label{eq:solve3}
G=(V\cup W, E)
\end{equation}

Onde \ref{eq:solve4}: 

\begin{equation} \label{eq:solve4}
V\cup W
\end{equation}


Compõem o conjunto de vértices e E o conjunto de arestas, ou seja, o subgrafo induzido por E e um grafo completo cujo conjunto de nós é V. 
V são todos os vértices que podem ser visitados e é composto pelo subconjunto T, que são os vértices que devem ser visitados por algum veiculo. W é um conjunto de vértices onde todos os M veículos devem passar. M é o numero de rotas de veículos que começam no vértice base V$_0$. 

O m-PRC atribui o conjunto de m rotas de veículos com as restrições: todas as m rotas de veículos começam e terminam na base V$_0$, Tem exatamente m rotas, cada vértice de V pertence a no máximo uma rota, cada vértice de T pertence a exatamente uma rota, com exceção a base, cada vértice de W deve ter uma rota que passa por ele e em uma distancia C de um vértice V visitado, O modulo da diferença entre o número de vértices de diferentes rotas não pode exceder um determinado valor R. A Figura \ref{fig:GrafoVUW} mostra o grafo da relação de V com W.

\begin{minipage}{\linewidth}
	\makebox[\linewidth]{
		\includegraphics[keepaspectratio=true,scale=0.5]{ibagens/grafoMPRC.png}}
	\captionof{figure}{Exemplo de gráfo não direcionado para V U W. \cite{Washington} }
	\label{fig:GrafoVUW}
\end{minipage}

Para utilizar o algoritmos genéticos com o modelo m-PRC, o trabalho propõem dois modelos. O AGS (Algoritmo genético sequencial), que utiliza heurísticas GENIUS e 2-opt balanceada para ajustes finais para tentar melhor a solução; O AGH(Algoritmos genéticos H-1-PRC), que utiliza heurísticas H-1-PRC-MOD e 2-opt balanceada em todo o processo de resolução.

A conclusão de \cite{Washington} é que a utilização de algoritmos genéticos para a resolução de de uma adaptação do problema de rotas de cobertura de veículos como bastante relevantes e de fácil manipulação. O modelo AGS resolve o problema de forma rápida e tem uma fácil implementação dentro dos critérios de comparação adotadas. O modelo AGH é mais lento e não conseguiu encontrar a solução para alguns exemplos.


