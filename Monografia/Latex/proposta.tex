
\chapter[Proposta]{Proposta}
%\addcontentsline{toc}{chapter}{Metodologia}
% ----------------------------------------------------------

A proposta deste trabalho é encontrar cenários no qual é viável a utilização de algoritmos genéticos para busca de caminhos, avaliar variados operadores de mutação e cruzamento assim como modificações em funções de aptidão visando a solução desse tipo de problema.

Iremos avaliar o GA em uma quantidade N de mapas, com tamanhos e padrões diferentes, comparando seus resultados com os algoritmos clássicos de busca A*, BFS e Dijkstra. 

Assim sendo, mapear os cenários e testar quando pode ser viável a utilização de GA, verificando qual tipo de ganho podemos ter nesses cenários. 

Os principais pontos de comparação que iremos focar são, consumo de memoria, tempo de processamento e tempo de execução.  



% A proposta deste trabalho é um modelo de busca de caminho do algoritmo BFS com IA em uma arquitetura paralela. O BFS tem sua utilização muito popular no ramo de jogos eletrônicos e existe uma necessidade na melhoria do desempenho destes algoritmos nesse meio. \cite{Ross_Graham}

% A primeira parte do modelo que iremos propor, trata-se da utilização de inteligência artificial, em especifico, algoritmos genéticos. Uma arquitetura híbrida foi demonstrada indicando sucesso na melhoria de desempenho. \cite{Ryan}

% Utilizar algoritmos genéticos de forma paralela mostra uma grande melhoria de desempenho conforme o numero de \textit{threads} é aumentado. \cite{Reza}

% Para a utilização em jogos eletrônicos é preciso que o modelo execute em tempo real, recalculando trajeto a cada iteração gráfica, tentando a partir daí, encontrar a uma solução mais próxima da ótima. Uma forma em tempo real do A* utilizando algoritmos genéticos teve resultados positivos. \cite{Ulysses2}
