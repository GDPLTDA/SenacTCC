\chapter{Metodologia}
O software tem que demostrar que serve para a situação proposta, com isso, diferentes rotas em situações onde é possível encontrar o caminho, ou não, foram criadas de forma a testar o software e evitar possíveis erros ou falsos positivos.
Utilizando a API do Google Maps como fonte de dados, informações reais de distância, tempo médio e localização são utilizadas para uma simulação mais próxima de uma situação real.
Por se tratar de entregas de pequenos porte, os testes foram criados com coordenadas a nível de cidade, São Paulo é a cidade para os testes, por se tratar de uma cidade com um alto índice de transito, a menor rota em metros pode não será a melhor escolha para aquele horário.
Todos os entregadores saem de uma única origem, antes de começar a entrega, é calculada uma rota geral, e ela é dividida de forma a entregar rotas possível para cada entregar, se o limite de entregados for ultrapassado, um avisa será emitido recomendando que deixe para o próximo dia as entregas mais distantes.
Cada destino tem um período permitido para entrega, e cada entrega demora no máximo 5 minutos para ser descarregada. Depois que uma entrega é feita, o software recalcula o próximo destino com base no transito atual e o período para entregar.
Se caso não é mais possível entregar no horário por causa do transito piorou, um alerta será emitido.

\chapter{Implementação}
 
Nesse capitulo será apresentado mais aprofundadamente as ferramentas e métodos que foram utilizados para a implementação do algoritmo genético para busca de rota com janela de tempo.
 
\section{O Projeto}
O software é separado em três projetos, todos utilizando .Net Core 2.0 com a linguagem C\# no Visual Studio 2017 para plataforma Windows ou \**nix.
O \textbf{PathFinder.Routes} nesse projeto estão os algoritmos de comunicação com o Google Maps e organização de rotas. 
O \textbf{PathFinder.GeneticAlgorithm} nesse projeto estão as implementações para a utilização do algoritmo genético.
O \textbf{PathFinder} projeto principal para inicialização do software e utiliza os dois outros projetos em sua implementação.

\section{Organização}
Os projetos são organizados utilizando o padrão do visual studio chamado de Solution.

\subsection{PathFinder}
\subsection{PathFinder.Routes}
\subsection{PathFinder.GeneticAlgorithm}


\section{Funcionalidades}

\section{Testes}