\chapter{Metodologia}
O software desenvolvido tem como objetivo demonstrar uma solução para o problema de PRVJT, descobrindo caminho viáveis entres diferentes endereços, onde é possível realizar as entregas no período limitado, também identificar roteiros que são impossíveis de serem realizados a tempo, e caminhos que serão necessários mais de um entregador para ser realizado. Alertas de não possibilidade de realizar a entrega será comunicado na interface do usuário.

Utilizando a API do Google Maps\cite{GoogleMatrix} como fonte de informações reais de distância, tempo médio e localizações. Utilizadas para uma simulação próxima de uma situação real de locomoção geográfica.

Por se tratar de entregas de pequenos porte, os testes foram criados com endereços dentro ou nas proximidades da cidade de São Paulo, um das maiores metrópoles do mundo, também tem um dos maiores índices de transito também segundo o TomTom Traffic Index \cite{TomTom}, a menor distância de rota pode não ser a melhor escolha para certos horários do dia, as vezes escolher uma rota com maior distância que evita transito primeiro é a melhor escolha para economizar tempo.

Para preparar o calculo da rota, deve-se levar em consideração que todos os entregadores partem de uma única origem que chamamos de deposito, a quantidade de entregadores é predefinida, e o sistema ira devolver rotas para atender todos os pontos visando diminuir a quantidade necessária de entregadores e distancia percorrida sem desrespeitar as restrições de janelas de tempo.

Existem situações onde não é possível entregar com numero de entregadores definido, essas situação podem depender da distancia, se for muito longe do deposito pode nao ter como chegar até o horário máximo do ponto de entrega, caso forem muito próximos qualquer mudança no transito torna impossível realizar a entrega, também a situação do numero limitado de entregadores, que não pode ser um número infinito, muitos endereços para poucos entregadores pode ser impossível encontrar uma definição de roteiro para realizar todas as entregas.

Cada endereço tem um horário de abertura e fechamento para realizar entregas, por exemplo, um super mercado recebe nos produtos na madrugada, já que receber em seu horário normal de abertura irá atrapalhar as compras dos clientes, então esse horário deve ser considerado para a escolha do próximo endereço. 
Depois que chega ao endereço se verifica o horário de abertura, se chegar antes, deve-se esperar até abrir, caso chegue antes do fechamento é considerado um tempo descarga. Caso algum entregador chegue depois de algum horário de fechamento é considerado que não foi encontrado uma solução viável para as restrições definidas.

Cada entrega também tem um tempo médio que demora para realizar a descarga do produto, produtos pequenos podem demorar minutos, muitos produtos demora mais para retirar do veiculo e produtos grande podem precisar ser levamos com mais demora.

Depois que uma entrega é feita, uma nova rota deve ser recalculada, agora como ponto inicial o endereço atual, todo o processo será refeito para verificar se o transito ou possível atrasos não afetaram a ordem dos próximos destinos, depois do recalculo o entregador deve seria o próximo destino indicado.

Se caso não for mais possível entregar no horário por motivos de piora de transito ou um grande tempo de atraso para descarregar, um alerta será emitido indicando que todos os destinos não podem ser visitados a tempo.
Cada entregador tem seu próprio recalculo de rota, sendo que se um entregador concluir todos os destinos, os outros continuam pedindo novas rotas até que todos terminem suas entregas.

O fluxograma a baixo demostra o funcionamento do software.

\begin{center}
    \makebox[\linewidth]{
        \includegraphics[keepaspectratio=true]{ibagens/Fluxograma.png}}
    \captionof{figure}{Fluxograma macro do funcionamento do software.  }
    \label{fig:FluxoSoftware}
\end{center}

Sempre que é necessário calcular a rota, o modulo de GA é chamado. Considerando que um individuo é uma rota completa, gera uma população de varias rotas, onde a ordem da rota é aleatória somente mantendo o deposito fixo como primeiro endereço para criar a população inicial.
Para cada rota da população de indivíduos a seleção determina dois para a realização do cruzamento, onde endereços das duas rotas são trocados de forma a criar duas novas rotas mantendo o deposito sempre como inicial. A mutação é executada individual em cada rota, mudando de posição um ou mais endereços da rota, mas sempre 
mantendo o deposito como ponto inicial.

Depois que todas as rotas dos indivíduos da população foram modificados, agora é hora de verificar quais são os melhores, o que define isso é a função de aptidão, todos os parâmetros da rota são agrupados em um único numero e a rota que tem o menor numero é a melhor rota da população. O valor de aptidão é definido com a soma da distância entre todos os endereços da rota, mais o tempo de cada um dos trajetos com o tempo de espera e descarga.

A função de aptidão do GA considera para uma frota homogênea de \textit{m} veículos o horário de saída como parâmetros inicial, com isso, utiliza o tempo dado pelo Google Maps entre os pontos e soma ao horário verificando se está dentro da janela de tempo do destino. Se o horário calculo for menor que o de abertura, é somado o tempo restante de espera e uma penalidade por chegar antes. Se o horário for maior que o tempo de fechamento, é somado o tempo restante de espera até a abertura no próximo dia mais uma penalidade pela diferença. O Valor de aptidão final é a soma da distancia em metros do percurso passando por todos os pontos, com o tempo total em minutos com as penalidades de chegar antes ou depois das janelas de tempo.

A função de aptidão é definida como:

\begin{center}
	\makebox[\linewidth]{
		\includegraphics[keepaspectratio=true]{ibagens/aptidao.png}}
	\captionof{figure}{Formula de aptidão baseado em \cite{Gendreau2} }
	\label{fig:MetodoAptidao}
\end{center}

Onde para cada ponto de entrega \textit{i} temos um \(e_i\)
e um \(li\)  que representam, respectivamente, a abertura e o fechamento
da janela de tempo do ponto \textit{i}, O \(tc_i\) é o tempo de chegada do veículo no ponto \(i\), \(te_i\) calculado como \(e_i - tc_i\). 

\(dk\) é a distância total percorrida na rota k para todo k=1,...,m, \(\alpha_i\)
é um coeficiente de penalidade associado à chegada do veículo no vértice i após o fechamento da janela de tempo.

O GA foi configurado com o numero de gerações em 200, tamanho da população em 1000, melhores indivíduos por geração em 20, probabilidade de cruzamento em 70\% e probabilidade de Mutação em 0,5\%.

Os testes foram configurados com um numero de 6 roteiros diferentes, endereços dentro da cidade de São Paulo e cidades próximas. Utilizando 4 tipos de mutação, sendo DisplacementMutation, InsertionMutation, InversionMutation e SwapMutation e um tipo de cruzamento, o SubRouteInsertionCrossover.

Considerando o transito médio enviado pelo Google Maps, para poder demostrar o impacto do transito nos caminhos calculados.
Tudo será rodado 10 vezes e será retirada uma média do valor de aptidão, por que, cada vez que roda o GA a resposta da solução pode mudar, por ele não ser determinístico.

E uma base pré-definidas rotas, para prevenir possíveis problemas sera ignorado o transito atual. Ja que o mesmo altera dependendo das condições do clima ou horário do dia. Então utilizando o Google Maps, um cache inicial foi preparado e o software utiliza simulando uma buscar ao Google Maps, com isso, a informação é obtida mais rapidamente e sempre fixa para garantir a resposta pré-determinada do teste.

O fluxograma a baixo demostra o funcionamento do software utilizando o GA.

\begin{center}
	\makebox[\linewidth]{
		\includegraphics[keepaspectratio=true]{ibagens/GA.png}}
	\captionof{figure}{Fluxograma macro da integração com o GA.}
	\label{fig:FluxoGA}
\end{center}

\chapter{Implementação}
 
Nesse capitulo será apresentado mais aprofundadamente as tecnologias, ferramentas e métodos que foram utilizados para a implementação do algoritmo genético para busca de rota com janela de tempo.

\section{Tecnologias}

O software foi desenvolvido na linguagem C\# com Visual Studio 2017 com o Framework  .Net Core 2.2 compatível com Windows, Linux e MacOs para o Back-End e ReactJs para a interface de utilização.
Toda comunicação da interface com o calculador de rotas é por WebAPi Rest, onde são trocados dados em formato de Json.

\section{Estrutura do Projeto}

Para facilitar o entendimento inicial do projeto, é importante ter uma orientação de onde está cada parte de seu comportamento. A Software é divido em 4 projetos principais, sendo um projeto feito em ReactJs e 3 em C\#, onde eles são:

\subsection{Calculador de Rotas- CalcRoute}

\subsection{Gerador de Dados - CalcRoute.RouteGenerate}

\subsection{API de Calculo de Rotas - CalcRoute.API}

\subsection{Interface para comunicação com a API - CalcRoute.Interface}

A interface é utilizada para interagir com o calculador de rotas de uma forma mais simples, podendo procurar por endereços, adicionar a lista, definir o deposito, os horários de cada endereço, numero de entregadores e tempo de descarga de cada endereço. Para melhor visualização dos testes, é possível encontrar todos os roteiros de teste para fácil execução e verifica os resultados.

\begin{center}
	\makebox[\linewidth]{
		\includegraphics[keepaspectratio=true,scale=0.5]{ibagens/Interface.png}}
	\captionof{figure}{Aparência da interface}
	\label{fig:Interface}
\end{center}


\begin{itemize}
	\item No campo procurar é preciso digitar o endereço para ser localizado no Google Maps, uma lista de possíveis escolhas aparece a baixo assim que começa a digitar, encontra o endereço e selecione.
	\item O mapa exibe o endereço selecionado.
	\item Abertura é o horário que que é permitida a entrada no local, no caso do deposito é o horário que o entregador sai para realizar as entregas.
	\item Fechamento é o horário limite para realizar a entrega, no caso do deposito, é o horário limite para terminar todas as entregas, horário que o deposito fecha.
	\item Entregadores é o numero de entregados disponível para realizar as entregas.
	\item Descarga o tempo médio para descarregar no endereço de entrega, no caso do deposito esse campo não é utilizado.
	\item Adicionar Agrupa os campos do endereço, abertura, fechamento e descarga para a lista.
	\item Lista de endereços reúne todos os endereços escolhidos para enviar para o calculador de rotas, o primeiro da lista é sempre o deposito
	\item Calcular Rota faz a chama da Api de calculo e envia a lista.
\end{itemize}

\begin{center}
	\makebox[\linewidth]{
		\includegraphics[keepaspectratio=true,scale=0.5]{ibagens/InterfaceResultado.png}}
	\captionof{figure}{Aparência da interface para exibição dos resultados}
	\label{fig:InterfaceResultado}
\end{center}

Cada entregador é exibido de forma separada, com sua própria tabela de endereço e recalculo para o próximo destino. A lista acima do mapa, é possível escolher qual rota será exibida no mapa, trocando entre os entregadores.
A tabela tem 8 colunas com informações de cada percurso, essa colunas são:
\begin{itemize}
	\item Endereço Saída: É o endereço que o entregador vai sair, na primeira vez será do deposito. Todos entregadores saem do mesmo endereço de deposito.
	\item Endereço Chegada: Endereço que o entregador chegará depois que sair do endereço de saída.
	\item KM: É a distancia em Quilômetros de cada percurso indicada pelo Google Maps.
	\item Saída: Horário que o entregador sairá para realizar a entrega.
	\item Percurso: Tempo de locomoção para chegar até o destino de entrega.
	\item Espera: Se chegar no destino antes do horário de abertura, esse é o tempo que o entregador ficará esperando.
	\item Entrada: Horário que o entregador consegui entrar para começar a realizar a descarga da entrega.
	\item Descarga: Tempo Médio para descarregador toda a encomenda.
\end{itemize}
Depois de finalizar a entrega o campo saída deve ser preenchido com o horário que está tudo pronto para fazer a próxima entrega e clicar no botão Próxima Rota, uma nova chamada para a Api será feita somente com o entregador pedido, esse processo pode ser repetido até que todas os endereços sejam visitados.
