\chapter{Conclusão}
Nesse capítulo é apresentado como os testes foram organizados, as limitações do software, futuras melhorias e os resultados encontrados.

\section{Testes}
Os testes foram gerados de forma automatizada para todos os parâmetros possíveis. A primeira etapa para a definição do que precisa ser testado foi a identificação de possíveis situações que o software não teria como dar uma resposta, ou seja, suas limitações. Identificamos 3 possíveis limitações ao rodar o software:  

\textbf{Não é possível entregar a tempo}: Quando as rotas são distribuídas aos entregadores, é possível aparecer uma situação onde a distribuição de rotas encontra uma rota que não é possível entregar no horário limite, então a roteiro não é possível com o numero de entregadores ou o tempo de locomoção passa do horário limite.

\textbf{Limite de Entregadores:} Um dos parâmetros iniciais é o numero de entregadores para realizar o roteiro, na distribuição de rotas se esse numero for excedido o rodeiro não é possível.

\textbf{Tempo limite de entrega excedido:} Depois que as rotas estão distribuídas entre os entregadores, a rota de cada entregador é recalcula, é possível que o transito fique pior depois que o entregador começou a entregar, fazendo com que o tempo limite seja excedido.

\subsection{Roteiros dos Testes}
Os roteiros foram escolhidos de forma arbitraria com endereços dentro ou próximos da cidade de São Paulo. As listas de endereços podem ser vista a baixo:

\begin{table}[h]
	\centering
	\caption{Senac}
	\label{Roteiro1}
	\begin{tabular}{C{3cm}C{8cm}C{2cm}C{2cm}}
		\toprule
		Nome                     & Endereço                                                         & Aber. & Fech. \\ \midrule
		Senac Largo Treze        & R. Dr. Antônio Bento, 393 - Santo Amaro                          & 09:00    & 12:00      \\
		Senac Taboão da Serra    & Rua Salvador Branco de Andrade, 182 - Jardim Sao Miguel          & 09:00    & 18:00      \\
		Senac Jabaquara          & Av. do Café, 298 - Jabaquara                                     & 09:00    & 11:00      \\
		Senac Osasco             & R. Dante Batiston, 248 - Centro                                  & 12:00    & 15:00      \\
		Senac Santana            & R. Voluntários da Pátria, 3167 - Santana                         & 10:00    & 19:00      \\
		Senac Tatuapé            & R. Cel. Luís Americano, 130 - Tatuapé                            & 15:00    & 19:00      \\
		Senac Vila Prudente      & Rua do Orfanato, 316 - Vila Prudente                             & 09:00    & 17:00      \\
		Senac - Campos do Jordão & Av. Frei Orestes Girardi, 3549 - Capivari, Campos do Jordão - SP & 09:00    & 17:00      \\ \bottomrule
	\end{tabular}
\end{table}

O teste da Tabela \ref{Roteiro1} começa o roteiro no Senac Nacões Unidas, endereço Av. Eng. Eusébio Stevaux, 823 - Santo Amaro, horário de saída 09:00, horário de volta 20:00:00, com 3 entregadores disponíveis e um tempo de espera médio em cada ponto 20 minutos.

\begin{table}[h]
	\centering
	\caption{Extra}
	\label{Roteiro2}
	\begin{tabular}{C{3cm}C{8cm}C{2cm}C{2cm}}
		\toprule
		Nome                     & Endereço                                                         & Aber. & Fech. \\ \midrule
		Extra Hipermercado       & R. João Batista de Oliveira, 47 - Centro, Taboão da Serra - SP     & 09:00    & 22:00      \\
		Extra João Dias          & Av. Guido Caloi, 25 - Jardim São Luís, São Paulo - SP              & 09:00    & 22:00      \\
		Extra Aeroporto          & Avenida Washignton Luís, 5859 - Jd. Aeroporto, São Paulo - SP      & 09:00    & 22:00      \\
		Extra - Itaim Bibi       & R. João Cachoeira, 899 - Itaim Bibi, São Paulo - SP                & 06:00    & 22:00      \\
		Extra - Ricardo Jafet    & Av. Dr. Ricardo Jafet, 1501 - Vila Mariana, São Paulo - SP         & 09:00    & 22:00      \\
		Extra                    & R. Nossa Sra. das Mercês, 29 - Vila das Merces, São Paulo - SP     & 09:00    & 22:00      \\
		Extra Hipermercado       & Av. Brigadeiro Luís Antônio, 2013 - Bela Vista, São Paulo - SP     & 09:00    & 22:00      \\
		Extra                    & Rua Três Rios, 282 - Bom Retiro, São Paulo - SP                    & 09:00    & 22:00      \\
		Extra Hiper Guarapiranga & Av. Guarapiranga, 752 - Socorro, São Paulo - SP                    & 09:00    & 22:00      \\
		Extra - Jardim Angela    & Estrada Velha do M'Boi Mirim, 4374 - Jardim Angela, São Paulo - SP & 09:00    & 22:00      \\
		Extra Hiper              & Av. Sen. Teotônio Vilela, 2926 - Jardim Iporanga, São Paulo - SP   & 09:00    & 22:00      \\ \bottomrule
	\end{tabular}
\end{table}

O teste da Tabela \ref{Roteiro2} começa o roteiro no Extra Morumbi, endereço Av. das Nações Unidas, 16741 - Santo Amaro, horário de saída 06:00, horário de volta 22:00:00, com 10 entregadores disponíveis e um tempo de espera médio em cada ponto 60 minutos.

\begin{table}[h]
	\centering
	\caption{MacDonald's}
	\label{Roteiro3}
	\begin{tabular}{C{3cm}C{8cm}C{2cm}C{2cm}}
		\toprule
		Nome                     & Endereço                                                         & Aber. & Fech. \\ \midrule
		McDonald's Augusta            & R. Augusta, 1856 - Cerqueira César, São Paulo - SP                                & 08:00    & 23:00      \\
		McDonald's Brigadeiro         & Av. Brigadeiro Luís Antônio, 3477/3481 - Jardim Paulista, São Paulo - SP          & 09:00    & 19:00      \\
		McDonald's José Maria         & Av. José Maria Whitaker, 81 - Jardim Paulista, São Paulo - SP                     & 09:00    & 21:00      \\
		McDonald's Nações Unidas      & Av. das Nações Unidas, 12555 - Pinheiros, São Paulo - SP                          & 06:00    & 21:00      \\
		McDonald's Eliseu de Almeida  & Av. Eliseu de Almeida, 2700 - Jardim Peri Peri, São Paulo - SP                    & 09:00    & 18:00      \\
		McDonald's Vital Brasil       & Av. Vital Brasil, 1256 - Butantã, São Paulo - SP                                  & 09:00    & 18:00      \\
		McDonald's Henrique Schaumann & Rua Henrique Schaumann, 80/124 - Cerqueira César, São Paulo - SP                  & 09:00    & 23:00      \\
		McDonald's Santo Antônio      & Av. Roque Petroni Júnior, 1089 - Chácara Santo Antônio (Zona Sul), São Paulo - SP & 09:00    & 23:00 \\ \bottomrule
	\end{tabular}
\end{table}

O teste da Tabela \ref{Roteiro3} começa o roteiro no McDonald's Jardim Paulista, endereço R. Pamplona, 734 - Jardim Paulista, São Paulo - SP, horário de saída 09:00, horário de volta 23:00:00, com 3 entregadores disponíveis e um tempo de espera médio em cada ponto 40 minutos.

\begin{table}[h]
	\centering
	\caption{Uninove}
	\label{Roteiro4}
	\begin{tabular}{C{3cm}C{8cm}C{2cm}C{2cm}}
		\toprule
		Nome                     & Endereço                                                         & Aber. & Fech. \\ \midrule
		Uninove Osasco                & R. Dante Batiston, 87 - Centro, Osasco - SP                                   & 08:00    & 23:00      \\
		Uninove Santo Amaro           & R. Amador Bueno - Santo Amaro, São Paulo - SP                                 & 08:00    & 23:00      \\
		Uninove São Bernardo do Campo & Av. Dom Jaime de Barros Câmara, 90 - Planalto, São Bernardo do Campo - SP     & 08:00    & 23:00      \\
		Uninove Vila Prudente         & Av. Professor Luiz Ignácio Anhaia Mello, 1363 - Vila Prudente, São Paulo - SP & 08:00    & 23:00      \\
		Uninove Vila Maria Baixa      & R. Itauna, 74 - Vila Maria Baixa, São Paulo - SP                              & 08:00    & 23:00      \\
		Uninove Barra Funda           & Av. Dr. Adolpho Pinto, 109 - Barra Funda, São Paulo - SP                      & 08:00    & 23:00      \\
		Uninove Mauá                  & R. Álvares Machado, 48 - Vila Bocaina, Mauá - SP                              & 08:00    & 23:00      \\
		Uninove Santo André           & R. Princesa Isabel - Vila Guiomar, Santo André - SP                           & 08:00    & 23:00 \\ \bottomrule
	\end{tabular}
\end{table}

O teste da Tabela \ref{Roteiro4} começa o roteiro no Uninove Vergueiro, endereço Rua Vergueiro, 235/249 - Liberdade, São Paulo - SP, horário de saída 08:00, horário de volta 23:00:00, com 5 entregadores disponíveis e um tempo de espera médio em cada ponto 90 minutos.

\begin{table}[h]
	\centering
	\caption{Pontos Turísticos de São Paulo}
	\label{Roteiro5}
	\begin{tabular}{C{3cm}C{8cm}C{2cm}C{2cm}}
		\toprule
		Nome                     & Endereço                                                         & Aber. & Fech. \\ \midrule
		Catedral Metropolitana de São Paulo & Praça da Sé - Sé, São Paulo - SP                                    & 00:00    & 23:59      \\
		Pte. Estaiada                       & Av. Jorn. Roberto Marinho, 85 - Cidade Monções, São Paulo - SP      & 00:00    & 23:59      \\
		Museu Catavento                     & Pq. Dom Pedro II - Av. Mercúrio, s/n - Brás, São Paulo - SP         & 09:00    & 16:00      \\
		Aquário de São Paulo                & R. Huet Bacelar, 407 - Ipiranga, São Paulo - SP                     & 09:00    & 17:00      \\
		Museu do Ipiranga                   & Parque da Independência - s/n - Ipiranga, São Paulo - SP            & 09:00    & 17:00      \\
		Parque Ibirapuera                   & Av. Pedro Álvares Cabral - Vila Mariana, São Paulo - SP             & 05:00    & 23:59      \\
		Zoológico De Sao Paulo              & Av. Miguel Estefno, 4241 - Vila Santo Estefano, São Paulo - SP      & 09:00    & 19:00      \\
		Jardim Botânico de São Paulo        & Av. Miguel Estefno, 3031 - Vila Água Funda, São Paulo - SP          & 09:00    & 17:00      \\
		Pateo do Collegio                   & Pç. Pateo do Collegio, 2 - Centro, São Paulo - SP                   & 09:00    & 16:30      \\
		Parque Estadual Alberto Löfgren     & R. do Horto, 931 - Horto Florestal, São Paulo - SP                  & 06:00    & 18:00      \\
		Parque Estadual do Jaraguá          & R. Antônio Cardoso Nogueira, 539 - Vila Chica Luisa, São Paulo - SP & 07:00    & 17:00      \\
		Autódromo de Interlagos             & Av. Sen. Teotônio Vilela, 261 - Interlagos, São Paulo - SP          & 07:00    & 17:00      \\
		Anhembi Sambadrome                  & Av. Olavo Fontoura, 1209 - Santana, São Paulo - SP                  & 07:00    & 17:00 \\ \bottomrule
	\end{tabular}
\end{table}

O teste da Tabela \ref{Roteiro5} simula a situação de 1 turista hospedado no Hotel Ibis São Paulo Paulista no endereço Av. Paulista, 2355 - Bela Vista, São Paulo - SP, em seu plano é conheço os endereços, permanecendo 60 minutos em cada, saindo do hotel 09:00 e voltando 23:00.

\section{Resultados}

O software se mostrou eficiente na organização do roteiro, identificando padrões não muito visíveis entre todos os destinos. O Google Maps ajuda na definição da rota por entregar o transito médio da rota fazendo com que o caminho escolhido pelo software fique mais próximo de uma situação real. O GA ajuda na escolhas das rotas por exemplo tentar minimizar o tempo e a distância, mesmo podendo demorar para encontrar uma solução ótima se o numero de gerações for muito alto, o tamanho da população ou numero de rotas. 

Utilizando um numero baixo de destinos o processo de recalculo de rotas para cada vez que chegar em um  destino, se mostra útil para identificar mudanças no transito e ainda chegar no horário proposto. Em casos que é preciso calcular para muitos destinos,é mais recomendado somente utilizar para uma divisão de tarefas ou pré-analise do roteiro, por tornar a analise muito demora, o tamanho da população e numero de gerações precisa ser mais altos para melhor precisão dos resultados.

Para comparativo foi utilizado os diferentes tipos de mutação e cruzamentos, de forma, a identificar uma melhor combinação para ser utilizada no problema. A tabela abaixo estão os resultados encontrados:

\begin{table}[h]
	\centering
	\caption{Comparação dos Cruzamentos e Mutações}
	\label{Comparacao}
	\begin{tabular}{C{2.5cm}C{1.5cm}C{1.5cm}C{1.5cm}C{1.5cm}C{1.5cm}C{1.5cm}}
		\hline
		\textbf{Extras.txt}               & \textbf{EM} & \textbf{DIVM} & \textbf{DM} & \textbf{IM} & \textbf{IVM} & \multicolumn{1}{l|}{\textbf{SM}} \\ \hline
		\multicolumn{1}{l|}{\textbf{OBX}} & 109560      & 109560        & 109560      & 109560      & 109560       & 109560                           \\
		\multicolumn{1}{l|}{\textbf{PBX}} & 109560      & 109560        & 109560      & 109560      & 109560       & 109560                           \\
		&             &               &             &             &              &                                  \\ \hline
		\textbf{MacDonalts.txt}           & \textbf{EM} & \textbf{DIVM} & \textbf{DM} & \textbf{IM} & \textbf{IVM} & \textbf{SM}                      \\ \hline
		\multicolumn{1}{l|}{\textbf{OBX}} & 43320       & 43320         & 43320       & 43320       & 43320        & 43320                            \\
		\multicolumn{1}{l|}{\textbf{PBX}} & 43320       & 43320         & 43320       & 43320       & 43320        & 43320                            \\
		&             &               &             &             &              &                                  \\ \hline
		\textbf{Senacs.txt}               & \textbf{EM} & \textbf{DIVM} & \textbf{DM} & \textbf{IM} & \textbf{IVM} & \textbf{SM}                      \\ \hline
		\multicolumn{1}{l|}{\textbf{OBX}} & 311876      & 311876        & 311876      & 311876      & 311876       & 311876                           \\
		\multicolumn{1}{l|}{\textbf{PBX}} & 311876      & 311876        & 311876      & 311876      & 311876       & 311876                           \\
		\textbf{}                         &             &               &             &             &              &                                  \\ \hline
		\textbf{Uninoves.txt}             & \textbf{EM} & \textbf{DIVM} & \textbf{DM} & \textbf{IM} & \textbf{IVM} & \textbf{SM}                      \\ \hline
		\multicolumn{1}{l|}{\textbf{OBX}} & 190656      & 190656        & 190656      & 190656      & 190656       & 190656                           \\
		\multicolumn{1}{l|}{\textbf{PBX}} & 190656      & 190656        & 190656      & 190656      & 190656       & 190656                           \\
		\textbf{}                         &             &               &             &             &              &                                  \\ \hline
		\textbf{Turisticos.txt}           & \textbf{EM} & \textbf{DIVM} & \textbf{DM} & \textbf{IM} & \textbf{IVM} & \textbf{SM}                      \\ \hline
		\multicolumn{1}{l|}{\textbf{OBX}} & 84573       & 84573         & 84573       & 84573       & 84573        & 84573                            \\
		\multicolumn{1}{l|}{\textbf{PBX}} & 84573       & 84573         & 84573       & 84573       & 84573        & 84573                           
	\end{tabular}
\end{table}

Como mostrado na tabela \ref{Comparacao}, a mutação ou cruzamento escolhida não influencia no resultado, apresentando o mesmo valor de aptidão para todas as combinações em nos diferentes roteiros.


\section{Trabalhos futuros}

Devido a média de tempo para se encontrar as rotas, avaliar a possível paralelização da rotina de algoritmos genéticos para tornar o projeto viável para uso em uma aplicação web.
Avaliar a aplicação de heurísticas  \(\lambda-interchange\), \textit{k-node}, \textit{Or-opt} no calculo do \textit{Fitness}.
Utilizar uma base de dados maior para os testes, baseada em casos reais de logística.
Criar uma aplicação web para utilização em um caso real de entregas.




