\chapter{Resultados}
Nesse capítulo é apresentado como os testes foram organizados, as limitações do software, os resultados encontrados e futuras melhorias.

\section{Comparações}

Implementamos para comparação 4 tipos de mutação, com duas definições de quantidade de gerações e tamanho de população, essas sendo, 1 - com uma população de 100 indivíduos e uma quantidade de 1000 gerações. e 2 - com uma população de 50 indivíduos e uma quantidade de 500 gerações.
 
Assim avaliando se é possível adquirir um resultado viável em menor tempo. 

Cada configuração foi executada 10 vezes, e depois calculado a média do valor de aptidão para evitar o efeito probabilístico do GA. Para isso implementamos um projeto que extrai essas informações, o Route.DataGeneration, todas os dados das possibilidades foram armazenados em um arquivo CSV e organizados na tabela abaixo. 
Para garantir que os valores de distancia e tempo entre dois pontos seja o mesmo para todos os testes, um cache das distancias entre os todos os pontos é reutilizado para cada um dos testes.

\begin{center}
	\makebox[\linewidth]{
		\includegraphics[keepaspectratio=true,scale=1.4]{ibagens/MutacaoComparacao.png}}
	\captionof{figure}{Comparação das Mutações, Menor é melhor}
	\label{fig:MutacaoComparacao}
\end{center}

A figura \ref{fig:MutacaoComparacao} demonstra a comparação entre as mutações, \textit{Inversion} e o \textit{Swap} não se demonstraram menos eficazes, Não sendo as melhores escolhas para criação das rotas. O \textit{Displacement} e o \textit{Insertion} conseguiram chegar em uma mesma média de aptidão para todas os arquivos, provando serem prováveis melhores escolhas para a busca de rotas.


\pagebreak
\section{Roteiros dos Testes}
Os roteiros foram escolhidos de forma arbitraria com endereços dentro ou próximos da cidade de São Paulo. Para garantir consistência entre os resultados para cada roteiro foi salvo um cache entre as distancias entre todos os pontos, de forma  ser utilizado o mesmo fator de distancia e tempo para todos os testes.

As listas de endereços e as rotas encontrada podem ser vista a baixo:

\subsection{Roteiro 1}

O primeiro roteiro (figura: \ref{fig:Roteiro1}) foi definido de forma que consiga ser realizado com apenas um entregador, o mesmo tem o seu deposito localizado em Jardim Paulista e realiza as entregas aos arredores da região da Vila Olímpia como pode ser observado na figura \ref{fig:Roteiro1-Mapa1}.
 
\begin{center}
	\makebox[\linewidth]{
		\includegraphics[keepaspectratio=true,scale=0.9]{../../Analise/Roteiro1/Tabela.png}}
	\captionof{figure}{Roteiro de um entregador}
	\label{fig:Roteiro1}
\end{center}

\begin{center}
	\makebox[\linewidth]{
		\includegraphics[keepaspectratio=true,scale=0.5]{../../Analise/Roteiro1/Mapa1.png}}
	\captionof{figure}{Mapa de um entregador}
	\label{fig:Roteiro1-Mapa1}
\end{center}



\subsubsection{500 Gerações e 50 População}

Como pode ser observado nas figuras (\ref{fig:Roteiro1-G1000P100-SwapMutation}, \ref{fig:Roteiro1-G1000P100-InversionMutation}, \ref{fig:Roteiro1-G1000P100-InsertionMutation}, \ref{fig:Roteiro1-G1000P100-DisplacementMutation}), tivemos resultados extremamente próximos, o que indica que com uma quantidade grande o suficiente de gerações os resultados convergem.

\begin{center}
	\makebox[\linewidth]{
		\includegraphics[keepaspectratio=true,scale=0.8]{../../Analise/Roteiro1/G1000P100/DisplacementMutation.png}}
	\captionof{figure}{Mutação DisplacementMutation}
	\label{fig:Roteiro1-G1000P100-DisplacementMutation}
\end{center}
\begin{center}
	\makebox[\linewidth]{
		\includegraphics[keepaspectratio=true,scale=0.8]{../../Analise/Roteiro1/G1000P100/InsertionMutation.png}}
	\captionof{figure}{Mutação InsertionMutation}
	\label{fig:Roteiro1-G1000P100-InsertionMutation}
\end{center}
\begin{center}
	\makebox[\linewidth]{
		\includegraphics[keepaspectratio=true,scale=0.8]{../../Analise/Roteiro1/G1000P100/InversionMutation.png}}
	\captionof{figure}{Mutação InversionMutation}
	\label{fig:Roteiro1-G1000P100-InversionMutation}
\end{center}
\begin{center}
	\makebox[\linewidth]{
		\includegraphics[keepaspectratio=true,scale=0.8]{../../Analise/Roteiro1/G1000P100/SwapMutation.png}}
	\captionof{figure}{Mutação SwapMutation}
	\label{fig:Roteiro1-G1000P100-SwapMutation}
\end{center}

\subsubsection{1000 Gerações e 100 População}

Dobrando a quantidade de gerações e população é possível de observar uma pequena melhoria no resultado (\ref{fig:Roteiro1-G500P50-SwapMutation}, \ref{fig:Roteiro1-G500P50-InversionMutation}, \ref{fig:Roteiro1-G500P50-InsertionMutation}, \ref{fig:Roteiro1-G500P50-DisplacementMutation}). Que provavelmente não justifique o tempo e processamento necessário parar se adquirir a resposta.

\begin{center}
	\makebox[\linewidth]{
		\includegraphics[keepaspectratio=true,scale=0.8]{../../Analise/Roteiro1/G500P50/DisplacementMutation.png}}
	\captionof{figure}{Mutação DisplacementMutation}
	\label{fig:Roteiro1-G500P50-DisplacementMutation}
\end{center}
\begin{center}
	\makebox[\linewidth]{
		\includegraphics[keepaspectratio=true,scale=0.8]{../../Analise/Roteiro1/G500P50/InsertionMutation.png}}
	\captionof{figure}{Mutação InsertionMutation}
	\label{fig:Roteiro1-G500P50-InsertionMutation}
\end{center}
\begin{center}
	\makebox[\linewidth]{
		\includegraphics[keepaspectratio=true,scale=0.8]{../../Analise/Roteiro1/G500P50/InversionMutation.png}}
	\captionof{figure}{Mutação InversionMutation}
	\label{fig:Roteiro1-G500P50-InversionMutation}
\end{center}
\begin{center}
	\makebox[\linewidth]{
		\includegraphics[keepaspectratio=true,scale=0.8]{../../Analise/Roteiro1/G500P50/SwapMutation.png}}
	\captionof{figure}{Mutação SwapMutation}
	\label{fig:Roteiro1-G500P50-SwapMutation}
\end{center}


\subsection{Roteiro 2}

O segundo roteiro (figura: \ref{fig:Roteiro2}) foi pensado de forma a ser necessário um minimo de 2 entregadores, de forma a um conjunto de pontos de entrega próximos a São Paulo e um ponto de entrega em Campos de Jordão.

Desta forma, pelo tempo necessário para se realizar esta entrega mais longe e voltar ao depósito, é obrigatório que seja demandado o uso de um entregador a mais. Como pode ser visualizado nos mapas que representam um resultado de rota (\ref{fig:Roteiro2-Mapa1},\ref{fig:Roteiro2-Mapa2}).

\begin{center}
	\makebox[\linewidth]{
		\includegraphics[keepaspectratio=true,scale=0.9]{../../Analise/Roteiro2/Tabela.png}}
	\captionof{figure}{Roteiro de dois entregadores}
	\label{fig:Roteiro2}
\end{center}
\begin{center}
	\makebox[\linewidth]{
		\includegraphics[keepaspectratio=true,scale=0.5]{../../Analise/Roteiro2/Mapa1.png}}
	\captionof{figure}{Mapa de um de dois entregadores}
	\label{fig:Roteiro2-Mapa1}
\end{center}
\begin{center}
	\makebox[\linewidth]{
		\includegraphics[keepaspectratio=true,scale=0.5]{../../Analise/Roteiro2/Mapa2.png}}
	\captionof{figure}{Mapa de um de dois entregadores}
	\label{fig:Roteiro2-Mapa2}
\end{center}


\subsubsection{500 Gerações e 50 População}

Para cada mutação os resultados convergiram da mesma forma que o primeiro, de forma a não ser possível identificar uma mudança significativa em sua aplicação.

Pode ser verificado nas figuras \ref{fig:Roteiro2-G500P50-SwapMutation}, \ref{fig:Roteiro2-G500P50-InversionMutation}, \ref{fig:Roteiro2-G500P50-InsertionMutation} e \ref{fig:Roteiro2-G500P50-DisplacementMutation} que o resultado divide de forma quase parcial a quantidade de pontos de entrega entre ambos os entregadores,  mesmo que um  entregador volte ao depósito na metade do dia e o outro entregador chegue ao final, pois o segundo tem sua entrega distante configurada no meio de seu roteiro, tendo assim entregas para serem realizados em seu caminho de ida e volta ao destino mais longe.

\begin{center}
	\makebox[\linewidth]{
		\includegraphics[keepaspectratio=true,scale=0.8]{../../Analise/Roteiro2/G500P50/DisplacementMutation/Rota1.png}}
	\makebox[\linewidth]{
		\includegraphics[keepaspectratio=true,scale=0.8]{../../Analise/Roteiro2/G500P50/DisplacementMutation/Rota2.png}}
	\captionof{figure}{Mutação DisplacementMutation}
	\label{fig:Roteiro2-G500P50-DisplacementMutation}
\end{center}
\begin{center}
	\makebox[\linewidth]{
		\includegraphics[keepaspectratio=true,scale=0.8]{../../Analise/Roteiro2/G500P50/InsertionMutation/Rota1.png}}
	\makebox[\linewidth]{
		\includegraphics[keepaspectratio=true,scale=0.8]{../../Analise/Roteiro2/G500P50/InsertionMutation/Rota2.png}}
	\captionof{figure}{Mutação InsertionMutation}
	\label{fig:Roteiro2-G500P50-InsertionMutation}
\end{center}
\begin{center}
	\makebox[\linewidth]{
		\includegraphics[keepaspectratio=true,scale=0.8]{../../Analise/Roteiro2/G500P50/InversionMutation/Rota1.png}}
	\makebox[\linewidth]{
		\includegraphics[keepaspectratio=true,scale=0.8]{../../Analise/Roteiro2/G500P50/InversionMutation/Rota2.png}}
	\captionof{figure}{Mutação InversionMutation}
	\label{fig:Roteiro2-G500P50-InversionMutation}
\end{center}
\begin{center}
	\makebox[\linewidth]{
		\includegraphics[keepaspectratio=true,scale=0.8]{../../Analise/Roteiro2/G500P50/SwapMutation/Rota1.png}}
	\makebox[\linewidth]{
		\includegraphics[keepaspectratio=true,scale=0.8]{../../Analise/Roteiro2/G500P50/SwapMutation/Rota2.png}}
	\captionof{figure}{Mutação SwapMutation}
	\label{fig:Roteiro2-G500P50-SwapMutation}
\end{center}


\subsubsection{1000 Gerações e 100 População}

No caso do roteiro de 2 entregadores aplicar mais força computacional, com mais populações não trouxe nenhum ganho. O algorítimo se demonstra convergindo bem rápido para uma solução viável (figuras: \ref{fig:Roteiro2-G1000P100-InversionMutation},  \ref{fig:Roteiro2-G1000P100-DisplacementMutation},  \ref{fig:Roteiro2-G1000P100-InsertionMutation},  \ref{fig:Roteiro2-G1000P100-SwapMutation}).

\begin{center}
	\makebox[\linewidth]{
		\includegraphics[keepaspectratio=true,scale=0.8]{../../Analise/Roteiro2/G1000P100/DisplacementMutation/Rota1.png}}
	\makebox[\linewidth]{
		\includegraphics[keepaspectratio=true,scale=0.8]{../../Analise/Roteiro2/G1000P100/DisplacementMutation/Rota2.png}}
	\captionof{figure}{Mutação DisplacementMutation}
	\label{fig:Roteiro2-G1000P100-DisplacementMutation}
\end{center}
\begin{center}
	\makebox[\linewidth]{
		\includegraphics[keepaspectratio=true,scale=0.8]{../../Analise/Roteiro2/G1000P100/InsertionMutation/Rota1.png}}
	\makebox[\linewidth]{
		\includegraphics[keepaspectratio=true,scale=0.8]{../../Analise/Roteiro2/G1000P100/InsertionMutation/Rota2.png}}
	\captionof{figure}{Mutação InsertionMutation}
	\label{fig:Roteiro2-G1000P100-InsertionMutation}
\end{center}
\begin{center}
	\makebox[\linewidth]{
		\includegraphics[keepaspectratio=true,scale=0.8]{../../Analise/Roteiro2/G1000P100/InversionMutation/Rota1.png}}
	\makebox[\linewidth]{
		\includegraphics[keepaspectratio=true,scale=0.8]{../../Analise/Roteiro2/G1000P100/InversionMutation/Rota2.png}}
	\captionof{figure}{Mutação InversionMutation}
	\label{fig:Roteiro2-G1000P100-InversionMutation}
\end{center}
\begin{center}
	\makebox[\linewidth]{
		\includegraphics[keepaspectratio=true,scale=0.8]{../../Analise/Roteiro2/G1000P100/SwapMutation/Rota1.png}}
	\makebox[\linewidth]{
		\includegraphics[keepaspectratio=true,scale=0.8]{../../Analise/Roteiro2/G1000P100/SwapMutation/Rota2.png}}
	\captionof{figure}{Mutação SwapMutation}
	\label{fig:Roteiro2-G1000P100-SwapMutation}
\end{center}


\subsection{Roteiro 3}

O terceiro roteiro (figura: \ref{fig:Roteiro3}) foi pensado de forma a ser necessário um minimo de 3 entregadores, todos os pontos ficam em São Paulo, porem com uma quantidade maior de pontos de entrega e tempo de descarga que as rotas anteriores.

Assim sendo, dois entregadores receberam de forma geral os pontos mais distantes do depósito e opostos entre si, e o terceiro  recebeu os ponto mais próximos do deposito   porem com tempo maior de descarga. (\ref{fig:Roteiro3-Mapa1}, \ref{fig:Roteiro3-Mapa2}, \ref{fig:Roteiro3-Mapa3}).


\begin{center}
	\makebox[\linewidth]{
		\includegraphics[keepaspectratio=true,scale=0.9]{../../Analise/Roteiro3/Tabela.png}}
	\captionof{figure}{Roteiro de três entregadores}
	\label{fig:Roteiro3}
\end{center}
\begin{center}
	\makebox[\linewidth]{
		\includegraphics[keepaspectratio=true,scale=0.5]{../../Analise/Roteiro3/Mapa1.png}}
	\captionof{figure}{Mapa de um de três entregadores}
	\label{fig:Roteiro3-Mapa1}
\end{center}
\begin{center}
	\makebox[\linewidth]{
		\includegraphics[keepaspectratio=true,scale=0.5]{../../Analise/Roteiro3/Mapa2.png}}
	\captionof{figure}{Mapa de um de três entregadores}
	\label{fig:Roteiro3-Mapa2}
\end{center}
\begin{center}
	\makebox[\linewidth]{
		\includegraphics[keepaspectratio=true,scale=0.5]{../../Analise/Roteiro3/Mapa3.png}}
	\captionof{figure}{Mapa de um de três entregadores}
	\label{fig:Roteiro3-Mapa3}
\end{center}


\subsubsection{500 Gerações e 50 População}

Para 3 entregadores ao contrario das rotas anterior é visível uma diferença entre a utilização da mutações, no caso as que alteram sub-rotas performaram pior (\ref{fig:Roteiro3-G500P50-InversionMutation}, \ref{fig:Roteiro3-G500P50-DisplacementMutation}) que as que fazem trocas simples entre pontos (\ref{fig:Roteiro3-G500P50-InsertionMutation}, \ref{fig:Roteiro3-G500P50-SwapMutation}). Isso se da pelo aumento de complexidade com mais entregadores, como a estrategia de cruzamento se guia pela distância entre pontos, grandes alterações no genoma podem mais facilmente resultar em resultados piores.

\begin{center}
	\makebox[\linewidth]{
		\includegraphics[keepaspectratio=true,scale=0.8]{../../Analise/Roteiro3/G500P50/DisplacementMutation/Rota1.png}}
	\makebox[\linewidth]{
		\includegraphics[keepaspectratio=true,scale=0.8]{../../Analise/Roteiro3/G500P50/DisplacementMutation/Rota2.png}}
	\makebox[\linewidth]{
		\includegraphics[keepaspectratio=true,scale=0.8]{../../Analise/Roteiro3/G500P50/DisplacementMutation/Rota3.png}}
	\captionof{figure}{Mutação DisplacementMutation}
	\label{fig:Roteiro3-G500P50-DisplacementMutation}
\end{center}
\begin{center}
	\makebox[\linewidth]{
		\includegraphics[keepaspectratio=true,scale=0.8]{../../Analise/Roteiro3/G500P50/InsertionMutation/Rota1.png}}
	\makebox[\linewidth]{
		\includegraphics[keepaspectratio=true,scale=0.8]{../../Analise/Roteiro3/G500P50/InsertionMutation/Rota2.png}}
	\makebox[\linewidth]{
		\includegraphics[keepaspectratio=true,scale=0.8]{../../Analise/Roteiro3/G500P50/InsertionMutation/Rota3.png}}
	\captionof{figure}{Mutação InsertionMutation}
	\label{fig:Roteiro3-G500P50-InsertionMutation}
\end{center}
\begin{center}
	\makebox[\linewidth]{
		\includegraphics[keepaspectratio=true,scale=0.8]{../../Analise/Roteiro3/G500P50/InversionMutation/Rota1.png}}
	\makebox[\linewidth]{
		\includegraphics[keepaspectratio=true,scale=0.8]{../../Analise/Roteiro3/G500P50/InversionMutation/Rota2.png}}
	\makebox[\linewidth]{
		\includegraphics[keepaspectratio=true,scale=0.8]{../../Analise/Roteiro3/G500P50/InversionMutation/Rota3.png}}
	\captionof{figure}{Mutação InversionMutation}
	\label{fig:Roteiro3-G500P50-InversionMutation}
\end{center}
\begin{center}
	\makebox[\linewidth]{
		\includegraphics[keepaspectratio=true,scale=0.8]{../../Analise/Roteiro3/G500P50/SwapMutation/Rota1.png}}
	\makebox[\linewidth]{
		\includegraphics[keepaspectratio=true,scale=0.8]{../../Analise/Roteiro3/G500P50/SwapMutation/Rota2.png}}
	\makebox[\linewidth]{
		\includegraphics[keepaspectratio=true,scale=0.8]{../../Analise/Roteiro3/G500P50/SwapMutation/Rota3.png}}
	\captionof{figure}{Mutação SwapMutation}
	\label{fig:Roteiro3-G500P50-SwapMutation}
\end{center}

\subsubsection{1000 Gerações e 100 População}

Assim como nos casos anteriores aumentar as quantidades de gerações e população não resultou em um resultado melhor como pode ser visto em ( \ref{fig:Roteiro3-G1000P100-DisplacementMutation},  \ref{fig:Roteiro3-G1000P100-InsertionMutation}, \ref{fig:Roteiro3-G1000P100-InversionMutation}, \ref{fig:Roteiro3-G1000P100-SwapMutation}).

De forma equivalente a versão de 500 gerações e 50 indivíduos as mutações simples demonstraram melhor resultado.

\begin{center}
	\makebox[\linewidth]{
		\includegraphics[keepaspectratio=true,scale=0.8]{../../Analise/Roteiro3/G1000P100/DisplacementMutation/Rota1.png}}
	\makebox[\linewidth]{
		\includegraphics[keepaspectratio=true,scale=0.8]{../../Analise/Roteiro3/G1000P100/DisplacementMutation/Rota2.png}}
	\makebox[\linewidth]{
		\includegraphics[keepaspectratio=true,scale=0.8]{../../Analise/Roteiro3/G1000P100/DisplacementMutation/Rota3.png}}
	\captionof{figure}{Mutação DisplacementMutation}
	\label{fig:Roteiro3-G1000P100-DisplacementMutation}
\end{center}
\begin{center}
	\makebox[\linewidth]{
		\includegraphics[keepaspectratio=true,scale=0.8]{../../Analise/Roteiro3/G1000P100/InsertionMutation/Rota1.png}}
	\makebox[\linewidth]{
		\includegraphics[keepaspectratio=true,scale=0.8]{../../Analise/Roteiro3/G1000P100/InsertionMutation/Rota2.png}}
	\makebox[\linewidth]{
		\includegraphics[keepaspectratio=true,scale=0.8]{../../Analise/Roteiro3/G1000P100/InsertionMutation/Rota3.png}}
	\captionof{figure}{Mutação InsertionMutation}
	\label{fig:Roteiro3-G1000P100-InsertionMutation}
\end{center}
\begin{center}
	\makebox[\linewidth]{
		\includegraphics[keepaspectratio=true,scale=0.8]{../../Analise/Roteiro3/G1000P100/InversionMutation/Rota1.png}}
	\makebox[\linewidth]{
		\includegraphics[keepaspectratio=true,scale=0.8]{../../Analise/Roteiro3/G1000P100/InversionMutation/Rota2.png}}
	\makebox[\linewidth]{
		\includegraphics[keepaspectratio=true,scale=0.8]{../../Analise/Roteiro3/G1000P100/InversionMutation/Rota3.png}}
	\captionof{figure}{Mutação InversionMutation}
	\label{fig:Roteiro3-G1000P100-InversionMutation}
\end{center}
\begin{center}
	\makebox[\linewidth]{
		\includegraphics[keepaspectratio=true,scale=0.8]{../../Analise/Roteiro3/G1000P100/SwapMutation/Rota1.png}}
	\makebox[\linewidth]{
		\includegraphics[keepaspectratio=true,scale=0.8]{../../Analise/Roteiro3/G1000P100/SwapMutation/Rota2.png}}
	\makebox[\linewidth]{
		\includegraphics[keepaspectratio=true,scale=0.8]{../../Analise/Roteiro3/G1000P100/SwapMutation/Rota3.png}}
	\captionof{figure}{Mutação SwapMutation}
	\label{fig:Roteiro3-G1000P100-SwapMutation}
\end{center}


\subsection{Roteiro 4}

O quarto roteiro (figura: \ref{fig:Roteiro4}) foi pensado de forma a ser necessário um minimo de 4 entregadores, todos os pontos de entrega ficam fora da cidade de São Paulo, de forma bem espaçada.

O ponto mais distante em Olímpia, acabou sendo passado para apenas um entregador, por ser muito longe e ter um tempo de descarga relativamentre alto, apenas seria possível que um entregador conseguisse realizar essa entrega e voltar a tempo. Os outros pontos por mais que distantes foram distribuídos entre os demais entregadores de forma quase uniforme, respeitando de forma que o mesmo entregador entregue entre pontos mais próximos ou que façam parte de seu caminho ate um terceiro ponto mais distante. (\ref{fig:Roteiro4-Mapa1},\ref{fig:Roteiro4-Mapa2}, \ref{fig:Roteiro4-Mapa3},\ref{fig:Roteiro4-Mapa4}).

\begin{center}
	\makebox[\linewidth]{
		\includegraphics[keepaspectratio=true,scale=0.9]{../../Analise/Roteiro4/Tabela.png}}
	\captionof{figure}{Roteiro de quatro entregadores}
	\label{fig:Roteiro4}
\end{center}
\begin{center}
	\makebox[\linewidth]{
		\includegraphics[keepaspectratio=true,scale=0.5]{../../Analise/Roteiro4/Mapa1.png}}
	\captionof{figure}{Mapa de um de quatro entregadores}
	\label{fig:Roteiro4-Mapa1}
\end{center}
\begin{center}
	\makebox[\linewidth]{
		\includegraphics[keepaspectratio=true,scale=0.5]{../../Analise/Roteiro4/Mapa2.png}}
	\captionof{figure}{Mapa de um de quatro entregadores}
	\label{fig:Roteiro4-Mapa2}
\end{center}
\begin{center}
	\makebox[\linewidth]{
		\includegraphics[keepaspectratio=true,scale=0.5]{../../Analise/Roteiro4/Mapa3.png}}
	\captionof{figure}{Mapa de um de quatro entregadores}
	\label{fig:Roteiro4-Mapa3}
\end{center}
\begin{center}
	\makebox[\linewidth]{
		\includegraphics[keepaspectratio=true,scale=0.5]{../../Analise/Roteiro4/Mapa4.png}}
	\captionof{figure}{Mapa de um de quatro entregadores}
	\label{fig:Roteiro4-Mapa4}
\end{center}

\subsubsection{500 Gerações e 50 População}

Todas as mutações convergiram para o mesmo valor neste caso, por mais que sejam rotas muito espaçadas em distancia, a quantidade de combinações validas para realizar todas as entregas a tempo não é alta (figuras: \ref{fig:Roteiro4-G500P50-InversionMutation}, \ref{fig:Roteiro4-G500P50-DisplacementMutation}, \ref{fig:Roteiro4-G500P50-InsertionMutation}, \ref{fig:Roteiro4-G500P50-SwapMutation}).

\begin{center}
	\makebox[\linewidth]{
		\includegraphics[keepaspectratio=true,scale=0.8]{../../Analise/Roteiro4/G500P50/DisplacementMutation/Rota1.png}}
	\makebox[\linewidth]{
		\includegraphics[keepaspectratio=true,scale=0.8]{../../Analise/Roteiro4/G500P50/DisplacementMutation/Rota2.png}}
	\makebox[\linewidth]{
		\includegraphics[keepaspectratio=true,scale=0.8]{../../Analise/Roteiro4/G500P50/DisplacementMutation/Rota3.png}}
	\makebox[\linewidth]{
		\includegraphics[keepaspectratio=true,scale=0.8]{../../Analise/Roteiro4/G500P50/DisplacementMutation/Rota4.png}}
	\captionof{figure}{Mutação DisplacementMutation}
	\label{fig:Roteiro4-G500P50-DisplacementMutation}
\end{center}
\begin{center}
	\makebox[\linewidth]{
		\includegraphics[keepaspectratio=true,scale=0.8]{../../Analise/Roteiro4/G500P50/InsertionMutation/Rota1.png}}
	\makebox[\linewidth]{
		\includegraphics[keepaspectratio=true,scale=0.8]{../../Analise/Roteiro4/G500P50/InsertionMutation/Rota2.png}}
	\makebox[\linewidth]{
		\includegraphics[keepaspectratio=true,scale=0.8]{../../Analise/Roteiro4/G500P50/InsertionMutation/Rota3.png}}
	\makebox[\linewidth]{
		\includegraphics[keepaspectratio=true,scale=0.8]{../../Analise/Roteiro4/G500P50/InsertionMutation/Rota4.png}}
	\captionof{figure}{Mutação InsertionMutation}
	\label{fig:Roteiro4-G500P50-InsertionMutation}
\end{center}
\begin{center}
	\makebox[\linewidth]{
		\includegraphics[keepaspectratio=true,scale=0.8]{../../Analise/Roteiro4/G500P50/InversionMutation/Rota1.png}}
	\makebox[\linewidth]{
		\includegraphics[keepaspectratio=true,scale=0.8]{../../Analise/Roteiro4/G500P50/InversionMutation/Rota2.png}}
	\makebox[\linewidth]{
		\includegraphics[keepaspectratio=true,scale=0.8]{../../Analise/Roteiro4/G500P50/InversionMutation/Rota3.png}}
	\makebox[\linewidth]{
		\includegraphics[keepaspectratio=true,scale=0.8]{../../Analise/Roteiro4/G500P50/InversionMutation/Rota4.png}}
	\captionof{figure}{Mutação InversionMutation}
	\label{fig:Roteiro4-G500P50-InversionMutation}
\end{center}
\begin{center}
	\makebox[\linewidth]{
		\includegraphics[keepaspectratio=true,scale=0.8]{../../Analise/Roteiro4/G500P50/SwapMutation/Rota1.png}}
	\makebox[\linewidth]{
		\includegraphics[keepaspectratio=true,scale=0.8]{../../Analise/Roteiro4/G500P50/SwapMutation/Rota2.png}}
	\makebox[\linewidth]{
		\includegraphics[keepaspectratio=true,scale=0.8]{../../Analise/Roteiro4/G500P50/SwapMutation/Rota3.png}}
	\makebox[\linewidth]{
		\includegraphics[keepaspectratio=true,scale=0.8]{../../Analise/Roteiro4/G500P50/SwapMutation/Rota4.png}}
	\captionof{figure}{Mutação SwapMutation}
	\label{fig:Roteiro4-G500P50-SwapMutation}
\end{center}

\subsubsection{1000 Gerações e 100 População}

Todas as mutações convergiram para o mesmo valor assim como as rotas anteriores, em comparação com resultado de menos gerações e indivíduos tivemos resultados ligeiramente melhores, aonde a rota de um dos entregador 1 da figura \ref{fig:Roteiro4-G500P50-DisplacementMutation} teve seu segundo roteiro alternado com entregador 4, ficando como mostra a figura \ref{fig:Roteiro4-G1000P100-DisplacementMutation}.

Isso demonstra que com mais tempo é possível adquirir melhores resultados, porem o tempo gasto tende a ser inversamente proporcional a melhoria de rota, de forma que tivemos que dobrar os recursos do sistema para adquirir um resultado otimizado em $6.3\%$.

\begin{center}
	\makebox[\linewidth]{
		\includegraphics[keepaspectratio=true,scale=0.8]{../../Analise/Roteiro4/G1000P100/DisplacementMutation/Rota1.png}}
	\makebox[\linewidth]{
		\includegraphics[keepaspectratio=true,scale=0.8]{../../Analise/Roteiro4/G1000P100/DisplacementMutation/Rota2.png}}
	\makebox[\linewidth]{
		\includegraphics[keepaspectratio=true,scale=0.8]{../../Analise/Roteiro4/G1000P100/DisplacementMutation/Rota3.png}}
	\makebox[\linewidth]{
		\includegraphics[keepaspectratio=true,scale=0.8]{../../Analise/Roteiro4/G1000P100/DisplacementMutation/Rota4.png}}
	\captionof{figure}{Mutação DisplacementMutation}
	\label{fig:Roteiro4-G1000P100-DisplacementMutation}
\end{center}
\begin{center}
	\makebox[\linewidth]{
		\includegraphics[keepaspectratio=true,scale=0.8]{../../Analise/Roteiro4/G1000P100/InsertionMutation/Rota1.png}}
	\makebox[\linewidth]{
		\includegraphics[keepaspectratio=true,scale=0.8]{../../Analise/Roteiro4/G1000P100/InsertionMutation/Rota2.png}}
	\makebox[\linewidth]{
		\includegraphics[keepaspectratio=true,scale=0.8]{../../Analise/Roteiro4/G1000P100/InsertionMutation/Rota3.png}}
	\makebox[\linewidth]{
		\includegraphics[keepaspectratio=true,scale=0.8]{../../Analise/Roteiro4/G1000P100/InsertionMutation/Rota4.png}}
	\captionof{figure}{Mutação InsertionMutation}
	\label{fig:Roteiro4-G1000P100-InsertionMutation}
\end{center}
\begin{center}
	\makebox[\linewidth]{
		\includegraphics[keepaspectratio=true,scale=0.8]{../../Analise/Roteiro4/G1000P100/InversionMutation/Rota1.png}}
	\makebox[\linewidth]{
		\includegraphics[keepaspectratio=true,scale=0.8]{../../Analise/Roteiro4/G1000P100/InversionMutation/Rota2.png}}
	\makebox[\linewidth]{
		\includegraphics[keepaspectratio=true,scale=0.8]{../../Analise/Roteiro4/G1000P100/InversionMutation/Rota3.png}}
	\makebox[\linewidth]{
		\includegraphics[keepaspectratio=true,scale=0.8]{../../Analise/Roteiro4/G1000P100/InversionMutation/Rota4.png}}
	\captionof{figure}{Mutação InversionMutation}
	\label{fig:Roteiro4-G1000P100-InversionMutation}
\end{center}
\begin{center}
	\makebox[\linewidth]{
		\includegraphics[keepaspectratio=true,scale=0.8]{../../Analise/Roteiro4/G1000P100/SwapMutation/Rota1.png}}
	\makebox[\linewidth]{
		\includegraphics[keepaspectratio=true,scale=0.8]{../../Analise/Roteiro4/G1000P100/SwapMutation/Rota2.png}}
	\makebox[\linewidth]{
		\includegraphics[keepaspectratio=true,scale=0.8]{../../Analise/Roteiro4/G1000P100/SwapMutation/Rota3.png}}
	\makebox[\linewidth]{
		\includegraphics[keepaspectratio=true,scale=0.8]{../../Analise/Roteiro4/G1000P100/SwapMutation/Rota4.png}}
	\captionof{figure}{Mutação SwapMutation}
	\label{fig:Roteiro4-G1000P100-SwapMutation}
\end{center}

\subsection{Alteração do tempo com base no transito}

A solução desenvolvida leva em consideração a densidade do trafego, utilizando de recursos do Google Mapas para escolher qual nível de gravidade do transito devera ser utilizado na simulação. Estas são três, \textbf{Média}, \textbf{Otimista} e \textbf{Pessimista}.

\subsubsection{Média}

Esta configuração utiliza o que a API do Google chama de \textit{Melhor Sugestão}, tentando encontrar a situação média melhor de transito com base no horário de entrega. 
Um exemplo pode ser visualizado na figura \ref{fig:RoteiroMedia}.

\begin{center}
	\makebox[\linewidth]{
		\includegraphics[keepaspectratio=true,scale=0.8]{../../Analise/Situacoes/Media.png}}
	\captionof{figure}{Rota gerada com a média do historico de transito}
	\label{fig:RoteiroMedia}
\end{center}
\subsubsection{Otimista}

Esta configuração de trafego implica em testar o tempo de entrega com o minimo de tempo de transição entre dois pontos. Com o palpite mais otimista de transito possível. 
Em comparação com o resultado da figura \ref{fig:RoteiroMedia} o resultado otimista da figura \ref{fig:RoteiroOtimista} volta ao deposito no final 20 minutos antes.

\begin{center}
	\makebox[\linewidth]{
		\includegraphics[keepaspectratio=true,scale=0.8]{../../Analise/Situacoes/Otimista.png}}
	\captionof{figure}{Rota gerada com o melhor tempo de cada rota}
	\label{fig:RoteiroOtimista}
\end{center}
\subsubsection{Pessimista}

Esta configuração de trafego implica em testar o tempo de entrega com o máximo de tempo de transição entre dois pontos. Ou seja, com pior trafego.
Em comparação com o resultado da figura \ref{fig:RoteiroMedia} o resultado pessimista da figura \ref{fig:RoteiroPessimista} volta ao deposito no final do dia com 20 minutos a mais de tempo.

\begin{center}
	\makebox[\linewidth]{
		\includegraphics[keepaspectratio=true,scale=0.8]{../../Analise/Situacoes/Pessimista.png}}
	\captionof{figure}{Rota gerada com o pior tempo de cada rota}
	\label{fig:RoteiroPessimista}
\end{center}



\subsection{Limitação de entregadores}

A figura \ref{fig:RoteiroNaoepossivelentregar} representa um roteiro que não é possível de ser atendido com apenas 1 entregador, já a solução desenvolvida pode receber como restrição a quantidade máxima de entregadores. Neste caso o resultado é conforme pode ser visto na figura \ref{fig:NaoPodeEntregar}.

\begin{center}
	\makebox[\linewidth]{
		\includegraphics[keepaspectratio=true,scale=1.1]{../../Analise/RoteiroNaoepossivelentregarcomapenas1.png}}
	\captionof{figure}{Roteiro Não é possível entregar com um entregador}
	\label{fig:RoteiroNaoepossivelentregarcomapenas1}
\end{center}

\begin{center}
	\makebox[\linewidth]{
		\includegraphics[keepaspectratio=true,scale=0.5]{ibagens/NaoPodeEntregar.png}}
	\captionof{figure}{Mensagem de impossibilidade de entrega}
	\label{fig:NaoPodeEntregar}
\end{center}


\subsection{Impossibilidade de entregar}

Em casos de entrega impossível como o roteiro da figura \ref{fig:RoteiroNaoepossivelentregar}, não importa a quantidade de entregadores ou configurações de GA, o resultado sempre sera "impossível entregar", como na figura \ref{fig:NaoPodeEntregar2}

\begin{center}
	\makebox[\linewidth]{
		\includegraphics[keepaspectratio=true,scale=1.0]{../../Analise/RoteiroNaoepossivelentregar2.png}}
	\captionof{figure}{Roteiro não é possível entregar}
	\label{fig:RoteiroNaoepossivelentregar}
\end{center}

\begin{center}
	\makebox[\linewidth]{
		\includegraphics[keepaspectratio=true,scale=0.5]{ibagens/NaoPodeEntregar.png}}
	\captionof{figure}{Mensagem de impossibilidade de entrega}
	\label{fig:NaoPodeEntregar2}
\end{center}



\section{Limitações}
Foi possível identificar possíveis situações em que o software não foi capaz de dar uma resposta, estes sendo:

\textbf{Não é possível entregar a tempo}: Levando em consideração o horário de saída e o horário limite para realizar todas as entregas, é possível que um percurso entre os endereços tem seu tempo de trajeto mais demorado que o tempo disponível para a realização de todas as entregas, com isso seria impossível entregar, mesmo com mais entregadores, com isso, o software não consegue definir uma rota por considerar a velocidade média das vias por onde ele passará.

\textbf{Limite de Entregadores:} Para pedir a definição de uma rota é preciso indicar quantos entregadores estão disponível para realizar as entregas, em algumas situações é possível que mesmo dividindo para todos os entregadores, mais entregadores seriam precisos para chegar a tempo em todos os endereços, com isso, o software não consegue definir uma rota.

\textbf{Tempo limite de entrega excedido:} Se a rota não tiver nenhum percurso muito demorado e também for possível determinar a divisão da rota principal entre o numero de entregadores, podemos encontrar outro problema, sempre que um entregador chega a um destino, é possível sofre atrasos na descarga, um maior tempo de espera ou até o transito piorando por causa de um acidente em uma via principal, por exemplo, isso muda o tempo dos próximos percursos, podendo elevar muito o tempo do trajeto.

É possível encontrar a situação que seria preciso mais entregadores para finalizar a entrega, que não é mais possível por que o entregador já está em transito, também pode encontrar um percurso completamente parado,  com isso, o software não consegue definir uma rota para finalizar as entregas.

\section{Conclusão}

A aplicação de uma abordagem evolutiva genérica para o PRVJT,se mostrou ser eficaz, uma vez que fomos capazes de encontrar rotas otimizadas para varias combinações de endereço de entrega. Além disso, os resultados mostram que este método é robusto e escalável. 

Fomos capazes de criar uma aplicação Web, com interface simples, para realizar os cálculos em tempo real, o que significa que o sistema consegue encontrar uma solução com uma velocidade relativamente alta, já que existe um tempo limite \textit{timeout} em requisições HTTP.

O Google Maps ajudou na definição da rota por entregar o transito médio da rota fazendo com que o caminho escolhido pelo software fique mais próximo de uma situação real. 

O GA ajudou na escolhas das rotas para minimizar o tempo, a distância e o número de entregadores, encontrando padrões difíceis de ser vistos, porém, para encontrar uma solução próxima do ótimo é preciso perder performance para calcular a resposta, aumentando o numero de gerações e população para encontrar a rota.

Utilizando um número baixo de destinos o processo de recalculo de rotas para cada vez que chegar em um  destino, se mostra útil para identificar mudanças no trânsito e ainda chegar no horário proposto.

\section{Trabalhos futuros}

Devido a média de tempo para se encontrar as rotas, avaliar a possível paralelização da rotina de algoritmos genéticos.

Aplicar mais uma restrição, como uma quantidade máxima de carga por entregador.

Desenvolver um aplicativo mobile para avançar as próximas rotas para cada entregador.

Utilizar uma base de dados maior para os testes, baseada em casos reais de logística.