
\chapter[Introdução]{Introdução}
%\addcontentsline{toc}{chapter}{Introdução}
% ----------------------------------------------------------

\section{Motivação}

No meio empresarial é essencial pensar na área logística, essa é a área que gerencia os recursos,
matérias-primas, componentes, equipamentos, serviços e informação necessária para execução e 
controle das atividades da empresa. Ela tem como foco orquestrar estes itens de forma a encontrar
a melhor condição de operação no menor tempo possível~\cite{DIAS}.

Um dos principais pontos dentro da logística é o transporte, onde chega a custar até 60% 
de seu custo total.\cite{RODRIGUES} Logo é de interesse das empresas conseguir minimizar o custo de escoamento de seus produtos.

Graças a sua importância no processo produtivo a logística se tornou um grande fator competitivo entre empresas.
Isso se deve ao fato que a cadeia de suprimento está relacionada com agregação de valores e disponibilidade dos seus bens e
serviços para os clientes, fornecedores da empresa e os demais interessados. Independe do lugar que o interessado esteja um serviço ou produto apenas tem valor quando 
ele está disponível para ser consumindo~\cite{TSUDA}.

No planejamento estratégico de logística o principal problema esta relacionado a roteirização de veículos~\cite{TSUDA} também conhecido como PRV, 
para encontrar a rota menos custosa, é necessário calcular as possíveis combinações de um determinado problema. Contudo, dependendo do numero de combinações pode requerer um
processamento muito elevado, levando muito tempo para encontrar a solução ótima. Esse problema se encaixa na categoria NP-Difícil~\cite{CUNHA}, Neste tipo de problema não existe
uma nenhum algoritmo conhecido que consiga resolvê-lo em tempo polinomial.

\section{Objetivos}

Desenvolver uma solução que resolva o problema de PRV utilizando a meta-heurística algoritmos genéticos. Um sistema capaz de calcular uma rota entre vários destinos 
levando em consideração restrições de tempo e notificando a quantidade de motoristas necessários para realizar todas as entregas ate uma data limite estipulada. Além de levar  
em consideração o tempo de transito entre estes pontos, permitindo que um motorista possa recalcular a sua rota para otimizar o tempo a qualquer momento.

\subsection{Objetivos Específicos}

Utilizar uma API de terceiros para adquirir informações sobre endereços ou pontos no mapa, alem do tempo de locomoção entre os pontos.
Implementar um algorítimo genético capas de minimizar a rota entre todos os pontos.
Introduzir janelas de tempo nas entregas e adaptar o algoritmo genético para levar essas janelas de tempo em consideração.
Definir fata/hora limite e dividir entrega em mais de um entregador para respeitar essa hora/data limite caso seja necessário.
Criar um aplicativo capaz de consultar e recalcular a rota.

\section{Método de trabalho}

Desenvolver uma implementação de algoritmos genéticos capaz de calcular rotas entre pontos geográficos.
Desenvolver uma aplicação Web capaz de receber uma quantidade N de pontos no mapa uma data limite, e a partir deles utilizar a implementação previa para calcular as rotas e 
notificar o utilizador.
Ter uma pagina onde um motorista possa recalcular sua rota em qualquer ponto.

\section{Organização do trabalho}
Este trabalho é dividido em 4 capítulos. O primeiro capitulo faz uma introdução geral do problema, descrever os objetivos e a motivação para a resolução do problema proposto.

O segundo capitulo trata do problema de forma separada, mostrando o que existe na literatura para uma possível solução. Também explica de forma mais detalhada o funcionamento de dois exemplos de busca heurística, demostrando uma aplicação em um trabalho da literatura e dos algoritmos genéticos, explicando seu funcionamento e aplicação na literatura.
O terceiro capitulo é a proposta apresentada para a criação deste trabalho.

