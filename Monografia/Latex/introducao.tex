\chapter[Introdução]{Introdução}

No meio empresarial é essencial pensar na área logística, essa é a área que gerência os recursos, matérias-primas, componentes, equipamentos, serviços e informações necessárias para execução e controle das atividades da empresa. 
Ela tem como foco orquestrar estes itens de forma a encontrar melhores condições de operação no menor tempo possível \cite{DIAS}.

\section{Motivação}

Um dos principais pontos dentro da logística é o transporte, onde chega a custar até 60\% de seu custo total \cite{RODRIGUES}.
Logo é de interesse das empresas conseguir minimizar o custo de escoamento em seus produtos.
Graças a sua importância no processo produtivo a logística se tornou um grande fator competitivo entre empresas.
Isso se deve ao fato que a cadeia de suprimento está relacionada com agregação de valores e disponibilidade dos seus bens e serviços para os clientes, fornecedores da empresa e os demais interessados. 

Independe do lugar que o interessado esteja um serviço ou produto apenas tem valor quando ele está disponível para ser consumido \cite{TSUDA}.
O grande crescimento populacional, a descentralização dos pontos de venda e o aumento da variedade de produtos tem provocado o crescimento e o aumento da complexidade da rede de distribuição de bens e serviços.

No planejamento estratégico de logística o principal problema esta relacionado a roteirização de veículos \cite{TSUDA} também conhecido como PRV. 
O PRV é baseado em definir um conjunto de rotas que será percorrido por veículos obedecendo que cada rota começa e termina no depósito, todo consumidor é visitado somente uma vez e a demanda total de qualquer rota não pode ultrapassar capacidade dos veículos para encontrar a rota menos custosa, é necessário calcular as possíveis combinações de um determinado problema , contudo, dependendo do numero de combinações pode requerer um processamento muito elevado, levando muito tempo para encontrar a solução ótima.

Como exemplo de aplicações podemos citar:
\begin{itemize}
	\item Entrega postal;
	\item Entrega em domicilio, de produtos comprados nas lojas de varejo ou pela internet;
	\item Distribuição de produtos dos centros de distribuição (CD) de atacadistas para lojas do varejo;
	\item Escolha de rotas para ônibus escolares ou de empresas;
\end{itemize}

Por exigir um alto esforço computacional faz com que é PRV pertença a classe dos problemas NP-difíceis, não sendo possível resolver em tempo polinomial, torna-se importante uma boa escolha do método para poder melhorar o tempo para um solução, onde vem sendo estudado pela grande variedade de problemas reais a ele associados.

O problema de roteamento de veículos com janela de tempo , o PRVJT, é uma ampliação do PRV, onde é considerado um intervalo para a entrega ser feita, fazendo com que a complexidade aumente e tornando mais próximo de casos reais, onde tem horário para entregar.

\section{Objetivos}

O trabalho tem objetivo de desenvolver uma solução que resolva o problema de PRVJV utilizando a meta-heurística algoritmos genéticos. 
O sistema será capaz de calcular uma rota entre vários destinos levando em consideração restrições de tempo e notificando a quantidade de motoristas necessários para realizar todas as entregas até uma data/hora limite estipulada, levando em consideração o tempo de transito entre estes pontos, permitindo que um motorista possa recalcular a sua rota para otimizar o tempo a qualquer momento.
Será calculada uma rota minimizando a distancia entre os pontos, o tempo do percurso e se é possível chegar no horário.
Não sendo possível cumprir o horário,o caminho dividido para mais entregadores, se o objetivo for impossível, é emitido um aviso sempre que acontecer.

\subsection{Objetivos Específicos}

\begin{itemize}
	\item Utilizar API do Google Maps para adquirir informações sobre endereços e pontos no mapas.
	\item Desenvolver um algorítimo genético capas de minimizar a rota entre todos os pontos considerando tempo e distância.
	\item Introduzir janelas de tempo nas entregas e adaptar algoritmo genético para levar janelas de tempo em consideração.
	\item Definir dia/hora limite e dividir entrega em mais de um entregador para respeitar essa hora/data limite. 
	\item Identificar quando não é possível calcular a rota respeitando a dia/hora limite.
	\item Criar um site para a interface de recalculo por entregador e visualização das rotas.
\end{itemize}

\section{Método de trabalho}
O problema será testando levando em consideração diferentes situações próximas de reais.
Simulando endereços para as entregas e horários em um  ambiente controlado que sabemos a melhor resposta e sendo colocado em situações impossíveis. 

\section{Organização do trabalho}
Este trabalho é dividido em 4 capítulos. O primeiro capitulo faz uma introdução geral do problema, descrever os objetivos e a motivação para a resolução do problema proposto.

O segundo capitulo trata do problema de forma separada, mostrando o que existe na literatura para uma possível solução. Também explica de forma mais detalhada o funcionamento de dois exemplos de busca heurística, demostrando uma aplicação em um trabalho da literatura e dos algoritmos genéticos, explicando seu funcionamento e aplicação na literatura.
O terceiro capitulo é a proposta apresentada para a criação deste trabalho.

