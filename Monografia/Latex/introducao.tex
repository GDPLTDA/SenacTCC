\chapter[Introdução]{Introdução}

No meio empresarial é comum pensar em logística, essa é a área que gerencia os recursos, matérias-primas, componentes, equipamentos, serviços, informações necessárias para execução e controle das atividades da empresa. Ela tem como foco orquestrar esses itens de forma a encontrar melhores condições de operação.

Uma atividade da logística que envolve grande valor de negócio é a distribuição eficiente de mercadorias ou serviços. Os custos de transporte representam uma parcela significativa do preço de muitos produtos. \cite{DIAS}


\section{Motivação}

Com o crescimento populacional, a descentralização dos pontos de venda e o aumento da variedade de produtos tem provocado o crescimento da complexidade da rede de distribuição de bens e serviços. O valor da logística se manifesta em termos de tempo e lugar, ou seja, disponibilidade. Sendo assim eles não terão valor, a menos que estejam em poder dos clientes quando (tempo) e onde (lugar) eles
pretendem consumi-los \cite{TSUDA}.

As decisões de transporte são parte fundamental da estratégia e planejamento logístico, com destaque para o problema de roteirização veículos (PRV), isso porque o transporte  pode representar de um a dois terços dos custos logísticos totais \cite{Ballou, RODRIGUES}.

O PRV é definido de forma que a partir de um conjunto de rotas que será percorrido por uma quantidade \textit{N} veículos, onde cada rota começa e termina em um depósito, e todos os endereços serão visitados somente uma vez. 

A identificação da ordem dos destinos, quando há um número elevado de endereços, se torna complexa por se tratar de um problema combinatório, onde é preciso avaliar todas as combinações para encontrar uma rota de menor tempo e distância \cite{RMKarp}.

Sendo assim PRV pode exigir um alto esforço computacional, pertencendo a classe dos problemas NP-difíceis, não pode ser solucionado em tempo polinomial, sendo uma forma de combinação da solução do problema do Caixeiro Viajante e do Problema da Mochila \cite{HUMBERTO}, por isso torna-se importante uma boa escolha do método a ser usado para sua solução.

Existem variações do problema PRV que adicionam mais complexidade para se adequar aos problemas reais de logística, uma delas é o problema de roteamento de veículos com janela de tempo, o PRVJT, assim como o PRV também pertence a classe NP-difíceis, nele deve-se considerar um intervalo de tempo para o atendimento dos consumidores nos locais das entregas a serem realizadas, por exemplo, não poderia realizar uma entrega para um cliente, em um horário, em que não poderia receber ou para empresas que só funcionem em horário comercial. 

Como exemplo de aplicações podemos citar:
\begin{itemize}
	\item Entrega postal;
	\item Entrega em domicílio de produtos comprados nas lojas de varejo ou pela internet;
	\item Distribuição de produtos dos centros de distribuição (CD) de atacadistas para lojas do varejo;
	\item Escolha de rotas para ônibus escolares ou de empresas;
\end{itemize}

Para se aproximar de uma situação mais real, deve-se levar em consideração que o trânsito das grandes cidades muda constantemente, e o tempo de percorrer uma certa distância dependendo do dia e horário da semana também muda, assim como acidentes, obras em vias e etc, tornando o trânsito uma variável importante para o cálculo da rota de entrega. Tendo destinos com horários de funcionamento delimitados, pode não existir uma rota que satisfaça as restrições de horário, tornando impossível de ser encontrado uma rota que passe por todos os destinos com apenas um entregador, somando o problema de se identificar a quantidade de entregadores necessária para realizar todas as entregas respeitando todas as restrições de horários partindo do depósito. 

Os atuais resultados encontrados na literatura referentes ao PRV e PRVJT comprovam que os algoritmos exatos restringem-se à resolução de problemas-teste com tamanho reduzido e janelas de tempo apertadas. Embora hoje possamos resolver problemas com um tamanho que seja ligeiramente maior que os de alguns anos atrás, o crescimento da capacidade dos computadores e da eficiência dos algoritmos está muito distante da curva exponencial representada por este problema. Pode-se dizer que os métodos exatos não são uma alternativa viável para situações onde a um número maior de consumidores, como ocorre na maioria dos casos reais \cite{Chabrier}.

Por isso a utilização de meta-heurísticas tais como algoritmos genéticos, podem conseguir resultados satisfatórios em menor tempo que as soluções exatas. \cite{BraysyAG}

\section{Objetivos}

Este trabalho tem como objetivo aplicar uma solução computacional para o PRVJT utilizando algoritmos genéticos para endereços reais.

Ao encontrar rotas com muitos destinos de forma a não conseguir realizar todas as entregas, estas serão separadas em rotas menores, cada rota deve ser percorrida por um entregador diferente, e todas tem como endereço inicial e final o depósito.

Também sendo possível determinar o número máximo de entregadores, horário de saída do depósito e horário máximo para realizar todas as entregas, caso a rota não seja possível com esse número limite entregados até o horário limite, o usuário será sinalizado.

O trânsito é considerado como alterador de tempo entre os endereços, fazendo com que a solução mude dependendo do dia da semana e horário. Todas as rotas são organizadas considerando o trânsito médio.

Esta solução irá visar a minimização da quantidade de entregadores necessários, distância total percorrida e tempo para realizar o percurso.

\subsection{Objetivos Específicos}

\begin{itemize}
	\item Realizar a integração com o Google Maps, considerando o trânsito utilizando o tempo médio entre os endereços.
	\item Aplicar uma solução de algoritmos genéticos para PRVJT 
	\item Implementar uma ferramenta para rodar o GA e salvar os resultados.
	\item Criação de uma interface web para definição dos destinos, indicação do depósito, exibição em tabelas das rotas calculadas e exibição de cada rota em um mapa interativo do Google Maps.
\end{itemize}


\section{Método de trabalho}

Diferentes instancias do problema serão criadas para a simulação computacional,  com estes testes já integrados programa, podendo ser escolhido e executado de uma maneira simples. Os endereços são reais, escolhidos em diferentes pontos no mapa e horários de abertura e fechamento são os indicados no Google Maps para cada endereço. 
Utilizando essa ambiente controlado situações impossíveis devem ser rejeitas.

\section{Organização do trabalho}
Este trabalho é dividido em 4 capítulos. 
O primeiro capitulo faz uma introdução geral do problema, com a descrição dos objetivos e a motivação para resolução do problema proposto.

O segundo capitulo trata do problema de forma separada, mostrando o que existe na literatura para uma possível solução. 
Também explica de forma mais detalhada o funcionamento das heurísticas e aplicações dos algoritmos genéticos para problemas semelhantes.

O terceiro capitulo é a proposta apresentada para a criação deste trabalho.

O quarto capitulo detalha implementação do programa e métodos utilizados para o seu funcionamento.

O quinto capitulo exibe os testes executados, resultados encontrados e futuras melhorias que podem ser adicionadas ao projeto.