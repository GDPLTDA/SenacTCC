
\chapter[Introdução]{Introdução}
%\addcontentsline{toc}{chapter}{Introdução}
% ----------------------------------------------------------

\section{Motivação}

No meio empresarial é essencial pensar na área logística, essa é a área que gerencia os recursos,
matérias-primas, componentes, equipamentos, serviços e informação necessária para execução e 
controle das atividades da empresa. Tem como foco orquestrar todos esses itens de forma a encontrar
a melhor condição de operação no menor tempo possível. \cite{DIAS}

um dos principais pontos dentro da logística é o transporte, onde chega a custar até 60% 
de seu custo total.\cite{RODRIGUES} Logo é de interesse das empresas conseguir minimizar o custo de escoamento de seus produtos.

Graças a sua importância no processo produtivo a logística se tornou um grande fator competitivo entre empresas.
Isso se deve ao fato que a cadeia de suprimento está relacionada com agregação de valores e disponibilidade dos seus bens e
serviços para os clientes, fornecedores da empresa e os demais interessados. um serviço ou produto apenas tem valor quando 
ele está disponível para ser consumindo independe do lugar que o interessado esteja \cite{TSUDA}.

\section{Objetivos}



\subsection{Objetivos Específicos}
-Utilizar API do Google Maps para adquirir informações sobre endereços ou pontos no mapas
-Desenvolver um algorítimo genético capas de minimizar a rota entre todos os pontos (caixeiro
-Introduzir janelas de tempo nas entregas e adaptar algoritmo genético para levar essas janelas de tempo em consideração
-Definir dia/hora limite e dividir entrega em mais de um entregador para respeitar essa hora/data limite
-Criar um aplicativo capaz de consultar e recalcular a rota


\section{Método de trabalho}

\section{Organização do trabalho}
Este trabalho é dividido em 4 capítulos. O primeiro capitulo faz uma introdução geral do problema, descrever os objetivos e a motivação para a resolução do problema proposto.

O segundo capitulo trata do problema de forma separada, mostrando o que existe na literatura para uma possível solução. Também explica de forma mais detalhada o funcionamento de dois exemplos de busca heurística, demostrando uma aplicação em um trabalho da literatura e dos algoritmos genéticos, explicando seu funcionamento e aplicação na literatura.
O terceiro capitulo é a proposta apresentada para a criação deste trabalho.

