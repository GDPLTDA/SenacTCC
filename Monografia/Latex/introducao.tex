\chapter[Introdução]{Introdução}

No meio empresarial é essencial pensar na área logística, essa é a área que gerência os recursos, matérias-primas, componentes, equipamentos, serviços e informação necessária para execução e controle das atividades da empresa. 
Ela tem como foco orquestrar estes itens de forma a encontrar melhores condições de operação no menor tempo possível \cite{DIAS}.

\section{Motivação}

Um dos principais pontos dentro da logística é o transporte, onde chega a custar até 60\% de seu custo total \cite{RODRIGUES}.
Logo é de interesse das empresas conseguir minimizar o custo de escoamento em seus produtos.
Graças a sua importância no processo produtivo a logística se tornou um grande fator competitivo entre empresas.
Isso se deve ao fato que a cadeia de suprimento está relacionada com agregação de valores e disponibilidade dos seus bens e serviços para os clientes, fornecedores da empresa e os demais interessados. 

Independe do lugar que o interessado esteja um serviço ou produto apenas tem valor quando ele está disponível para ser consumindo \cite{TSUDA}.
O grande crescimento populacional, a descentralização dos pontos de venda e o aumento da variedade de produtos tem provocado o crescimento e o aumento da complexidade da rede de distribuição de bens e serviços.

No planejamento estratégico de logística o principal problema esta relacionado a roteirização de veículos \cite{TSUDA} também conhecido como PRV. 
O PRV é baseado em definir um conjunto de rotas que será percorrido por veículos obedecendo que cada rota começa e termina no depósito, todo consumidor é visitado somente uma vez e a demanda total de qualquer rota não pode ultrapassar capacidade dos veículos para encontrar a rota menos custosa, é necessário calcular as possíveis combinações de um determinado problema , contudo, dependendo do numero de combinações pode requerer um processamento muito elevado, levando muito tempo para encontrar a solução ótima.

Como exemplo de aplicações podemos citar:
\begin{itemize}
	\item Entrega postal;
	\item Entrega em domicilio, de produtos comprados nas lojas de varejo ou pela internet;
	\item Distribuição de produtos dos centros de distribuição (CD) de atacadistas para lojas do varejo;
	\item Escolha de rotas para ônibus escolares ou de empresas;
\end{itemize}

O problema de roteamento de veículos com janela de tempo , o PRVJT, é uma ampliação do PRV, onde é considerado um intervalo para a entrega ser feita.
Uma vez que o PRVJT exige um alto esforço computacional, pois pertence a classe dos problemas NP-difíceis, torna-se importante uma boa escolha do método usado para sua solução que consiga resolvê-lo em tempo polinomial.
Desde então, o problema vem sendo muito estudado, principalmente por sua alta complexidade e pela grande variedade de problemas reais a ele associados.

\section{Objetivos}

O trabalho tem objetivo de desenvolver uma solução que resolva o problema de PRVJV utilizando a meta-heurística algoritmos genéticos. Um sistema capaz de calcular uma rota entre vários destinos 
levando em consideração restrições de tempo e notificando a quantidade de motoristas necessários para realizar todas as entregas ate uma data limite estipulada. Além de levar  
em consideração o tempo de transito entre estes pontos, permitindo que um motorista possa recalcular a sua rota para otimizar o tempo a qualquer momento.

\subsection{Objetivos Específicos}

\begin{itemize}
	\item Utilizar API do Google Maps para adquirir informações sobre endereços ou pontos no mapas.
	\item Desenvolver um algorítimo genético capas de minimizar a rota entre todos os pontos (caixeiro viajante).
	\item Introduzir janelas de tempo nas entregas e adaptar algoritmo genético para levar essas janelas de tempo em consideração.
	\item Definir dia/hora limite e dividir entrega em mais de um entregador para respeitar essa hora/data limite.
	\item Criar um aplicativo capaz de consultar e recalcular a rota.
	\item Utilizar uma API de terceiros para adquirir informações sobre endereços ou pontos no mapa, alem do tempo de locomoção entre os pontos.
	\item Implementar um algorítimo genético capas de minimizar a rota entre todos os pontos.
	\item Introduzir janelas de tempo nas entregas e adaptar o algoritmo genético para levar essas janelas de tempo em consideração.
	\item Definir fata/hora limite e dividir entrega em mais de um entregador para respeitar essa hora/data limite caso seja necessário.
	\item Criar um aplicativo capaz de consultar e recalcular a rota.
\end{itemize}

\section{Método de trabalho}

 

\section{Organização do trabalho}
Este trabalho é dividido em 4 capítulos. O primeiro capitulo faz uma introdução geral do problema, descrever os objetivos e a motivação para a resolução do problema proposto.

O segundo capitulo trata do problema de forma separada, mostrando o que existe na literatura para uma possível solução. Também explica de forma mais detalhada o funcionamento de dois exemplos de busca heurística, demostrando uma aplicação em um trabalho da literatura e dos algoritmos genéticos, explicando seu funcionamento e aplicação na literatura.
O terceiro capitulo é a proposta apresentada para a criação deste trabalho.

