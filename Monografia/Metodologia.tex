\chapter[Metodologia]{Metodologia}

Desenvolvemos uma ferramenta para geração de mapas de forma automática, aonde escolhemos se queremos um mapa totalmente aleatório, o tamanho minimo do caminho e se gostaria de aplicar um padrão de repetição na criação do mapa.

O tamanho dos mapas gerados foi de 100x100, para os mapas com padrão utilizamos um padrão de repetição de tamanho 5x5.

Foram gerados 800 mapas para realizar a coleta de dados, sendo uma metade mapas com padrão e outra sem,
cada metade se divide em mais 4 aonde dessas 100 mapas são com solução para diagonal "Nunca", 100 para "Apenas sem obstáculos", 100 para "No máximo um obstaculo" e 100 para "Sempre".

Os algoritmos de busca que utilizamos na ferramenta são A\*, BFS, Dijkstra, IDA\*, todos foram rodados pelo uma vez para cada heurística selecionadas para testes, essas são Manhattam, Euclideana, Octil e Chebyshev.

Para o GA utilizamos uma função de aptidão baseada nas heurísticas implementadas, uma sem nenhuma alteração, outra penalizando caminhos cíclicos, outra penalizando encontro com paredes e mais uma penalizando tanto caminhos quanto encontro com paredes. 
Para operação de seleção utilizamos o algoritmo de Roleta, para operação de cruzamento utilizamos os Simples, OBX, PBX e para o operador de mutação foram implementados Bitwise, DIVM, DM, EM, IM,IVM, SM.
O GA foi rodado 10 vezes para cada configuração possível, levando em consideração heurística, aptidão, seleção e cruzamento. Depois utilizamos a media dos valores para comparar com algoritmos clássicos de busca

Coletamos os resultados de tempo, nós abertos e tamanho do caminho obtido para cada algoritmo de busca.
Depois realizamos uma comparação entre o comportamento de cada algorítimo, heurística e tipo de mapa. 


\chapter[Implementação]{Implementação}

Nesse capitulo será apresentado mais aprofundadamente as ferramentas e métodos que foram utilizados para realizar os testes do modelo proposto.

\section{Tecnologias}
O projeto é separado em Pathfinder e Pathfinder.UI foram desenvolvidos na linguagem C\# utilizando o .NET Standard Library 1.6. Pathfinder estão todas a =s implementações dos algoritmos de busca que temos para a comparação. Pathfinder.UI tem como objetivo exibir e utilizar o projeto Pathfinder. Ambos rodam em sistemas Windows e Unix utilizando o .Net Core 1.1 para a compilação.

\section{Estrutura do Projeto}

Essa seção tem como objetivo descrever como foram implementados os algoritmos 

\subsection {PathFinder}

Projeto de implementação de algoritmos de busca.
Sua estrutura de pastas está organizada em Abstraction, Constants, Core, Factories, Finders, GeneticAlgorithm, Heuristics e MapGenerators.

O \textbf{'project.csproj'} é o arquivo onde é definido as bibliotecas utilizadas e a versão do .NET Framework, as outras pastas agregam arquivos com informações relevantes a nossa implementação.

\subsection{Abstraction}	

Nesta pasta estão todos os arquivos a nível de abstração dos algoritmos de busca, esses são:

\textbf{IFactory}: Essa interface tem como objetivo padronizar as "fabricas", ferramentas que decidir e instanciar toda dependência necessária.

\textbf{IMap}: Essa interface tem como objetivo abstrair o comportamento da classe de mapa utilizada nos arquivos de busca, assim sendo por padrão todo algoritmo espera uma implementação de IMap para rodar.

\textbf{IHeuristic}: Essa interface abstrai o comportamento das heurísticas.

\textbf{IMapGenerator}: Essa interface tem como objeto abstrair os gerador de mapas.

\textbf{IFinder}: Essa interface é a responsável por abstrair todo comportamento dos algoritmos de busca.

\textbf{IGeneticAlgorithm}: Essa interface herda de IFinder, ela compartilha a mesma assinatura de métodos, propriedades e eventos, porem acrescenta a abstração necessária para
o utilização de GA.

\subsection{Constants}

Nesta pasta são listados arquivos de constantes e enumeradores.

\textbf{DiagonalMovement}: Lista as opções de diagonais na movimentação.

\textbf{DirectionMovement}: Lista as oito opções possíveis de se locomover a partir de um ponto para seus vizinhos 
(imagem)(cima, baixo, esquerda, direita, esquerda cima, esquerda baixo, direita cima, direita baixo).

\subsection{Core}

Nesta pasta são definidos as implementações e configurações bases.

\textbf{Container}: Esta classe é responsável por registar e resolver as implementações conhecidas das interfaces.

\textbf{Enumerators}: Contem as definições de enumerações, usados para usar nomes bem definidos ao invés de números avulsos no código.

\textbf{Extensions}: Arquivo com métodos auxiliares de lista para comportamento de uma estrutura de pilha.

\textbf{FileTools}: Classe responsável por toda manipulação de I/O de arquivos

\textbf{Map}: Implementação do IMap, tem como objetivo ser a estrutura de mapa base dos algoritmos de busca.

\textbf{Node}: Classe responsável por ser a representação de uma célula no mapa, ou seja, o mapa é uma matriz de \textbf{\textit{"Node"}}.

\textbf{Settings}: Contém toda configuração estática do projeto, do qual é carregado de um arquivo Json chamado "appsettings.json"

\subsection{Factories}

Nesta Pasta temos os arquivos responsáveis pelo instanciar as implementações de interfaces.

\textbf{FinderFactory}: Classe responsável por decidir e instanciar uma implementação IFinder.

\textbf{HeuristicFactory}: Classe responsável por decidir e instanciar uma implementação IHeuristic.

\textbf{MapGeneratorFactory}: Classe responsável por decidir e instanciar uma implementação IMapGenerator.

\subsection{Finders}

Nesta pasta temos definidas as implementações de todos os algoritmos de busca de caminho.

\textbf{AStarFinder}: Implementação do algoritmo de busca de caminho A* implementada em cima da interface IFinder.

\textbf{BestFirstSearchFinder}: Implementação do algoritmo de busca de caminho “Best First Search” implementada em cima da interface IFinder.

\textbf{DijkstraFinder}: Implementação do algoritmo de busca de caminho Dijkstra implementada em cima da interface IFinder.

\textbf{IDAStarFinder}: Implementação do algoritmo de busca de caminho IDA* implementada em cima da interface IFinder.

\textbf{GAFinder}: Implementação de um algoritmo genético para busca de caminhos implementada em cima da interface IFinder e IGeneticAlgorithm.

\subsection{Heuristics}

Nesta pasta são definidas as implementações de IHeuristic, responsáveis pelos cálculos de heurística.

\textbf{Manhattan}: Implementação da classe Manhattan implementada em cima da interface IHeuristic responsável por calcular a distancia Manhattam.

\textbf{Euclidean}: Implementação da classe Euclidean implementada em cima da interface IHeuristic responsável por calcular a distancia Euclideana.

\textbf{Octile}: Implementação da classe Octile implementada em cima da interface IHeuristic responsável por calcular a distancia Octile.

\textbf{Chebyshev}: Implementação da classe Chebyshev implementada em cima da interface IHeuristic responsável por calcular a distancia Chebyshev.

\section{Genetic Algorithm}

Nesta pasta são definidos todas as implementações referentes ao algoritmo genético, pela complexidade.
do algoritmo ele possui uma estrutura própria de pastas para definições e configurações de injeção de dependência.

\subsection{Abstraction}

Nesta pasta estão todos os arquivos a nível de abstração das etapas do algoritmo genético.

\textbf{ISelection}: Interface é responsável por abstrair os algoritmos de seleção.

\textbf{IGenome}: Interface tem como funcionalidade abstrair a definição de genoma.

\textbf{IFitness}: Interface tem como objetivo abstrair o calculo de fitness.

\textbf{IMutate}: Interface tem como objetivo abstrair os operadores de mutação.

\textbf{ICrossover}:  Interface tem como objetivo abstrair os operadores de cruzamento.

\textbf{IRandom}: Interface tem como objetivo abstrair a implementação de geração de números aleatórios.

\textbf{AbstractMutate}: Implementação base para operador de mutação.

\textbf{AbstractCrossover}: Implementação base para operador de cruzamento.

\subsection{Core}

\textbf{Adaptation}: Classe responsável para realizar a adaptação de um indivíduo novo após ser gerado.

\textbf{Enumerators}: Contem as definições de enumerações, usados para usar nomes bem definidos ao invés de números avulsos no código.

\textbf{GARandom}: Implementação responsável por gerar números aleatórios, implementa IRandom.

\textbf{GASettings}: Arquivo responsável por carregar configuração estática de GA, carrega do arquivo "GASettings.json”.

\textbf{Genome}: Classe responsável por representar o genoma no algoritmo de GA, implementa a IGenome.


\subsection{Selection}

Nesta pasta estão todas as implementações dos algoritmos de seleção.

\textbf{SelectionRandom}: Implementação de seleção de indivíduos aleatório.
\textbf{SelectionRouletteWheel}: Implementação de seleção roleta.

\subsection{Crossover}

Nesta pasta estão todas as implementações dos algoritmos de cruzamento.

\textbf{CrossoverOBX}: Implementação do operador de cruzamento OBX.

\textbf{CrossoverPBX}: Implementação do operador de cruzamento PBX.

\textbf{CrossoverSimple}:  Implementação do operador de cruzamento simples.

\subsection{Mutation}

Nesta pasta estão todas as implementações dos algoritmos de mutação.

\textbf{MutateBitwise}: Implementação do operador de cruzamento Bitwise.

\textbf{MutateDIVM}: Implementação do operador de cruzamento DIVM.

\textbf{MutateDM}: Implementação do operador de cruzamento DM.

\textbf{MutateEM}: Implementação do operador de cruzamento EM.

\textbf{MutateIM}: Implementação do operador de cruzamento IM.

\textbf{MutateIVM}: Implementação do operador de cruzamento IVM.

\textbf{MutateSM}: Implementação do operador de cruzamento SM.


\section{Projeto de UI}

Foi desenvolvido um projeto com objetivo de consumir a biblioteca de busca de caminhos, e poder visualiza-los.

\subsection{Abstraction}

Nesta pasta estão todos os arquivos a nível de abstração.

\textbf{IAppMode}: Abstração que define de que forma o app ira rodar.

\textbf{IViewer}: Abstração do tipo de visualizador.

\subsection{AppMode}

Pode-se configurar diferentes modos de rodar os algoritmo, nesse pasta estão a implementação das diferentes formas.

\textbf{SingleRunMode}: O programa será executado e rodara uma vez usando as configurações do arquivo estático.

\textbf{DynamicMode}: O programa ira perguntar qual algoritmo, heurística, tipo de diagonal, forma de visualização e cada operador do GA para rodar.

\textbf{BatchMode}: O software ira rodar N vezes cada algoritmo selecionado no arquivo de configuração, onde N também é definido neste arquivo, ao final ira salvar os resultados e cada mapa numa pasta na raiz do projeto.


\subsection{Core}

Nesta pasta são definidos as implementações e configurações bases.

\textbf{Enumerators}: Contem as definições de enumerações, usados para usar nomes bem definidos ao invés de números avulsos no código.

\textbf{RegisterConfig}: Neste arquivo são configurados os binds do visualizador para injeção de dependência.

\textbf{Settings}: Onde são carregados as configurações estáticas do arquivo "appsettings.json", neste são configurações da forma de visualização e do Batch.

\subsection{Factories}

Nesta Pasta temos os arquivos responsáveis pelo instanciar as implementações de interfaces.

\textbf{AppModeFactory}: Classe responsável por decidir e instanciar uma Implementação de IAppMode.

\textbf{ViwerFactory}: Classe responsável por decidir e instanciar uma Implementação de IViewer.

\subsection{Viewer}

Nesta pasta estão as diferentes formas de exibir os resultados das buscas.

\textbf{ConsoleViewer}: Classe responsável por apresentar a busca de caminhos em ASC no Console da aplicação.

\textbf{OpenGLViewer}: Classe responsável por apresentar a busca em uma janela em OpenGL.

\textbf{OpenGLWindow}: Classe que é utilizada pela OpenGLViewer para mostrar a janela com uma grid que mostra o andamento dos algoritmos.

\section{Estrutura do GA}

A busca utilizando o GA, segue com as operações base de todo GA, que são seleção, cruzamento, adaptação e mutação. 
Para cada interação é gerada uma população, a função de aptidão é calculada para cada indivíduo da população, 
desses o indivíduo com o valor mais próximo de zero é selecionado como o melhor.

\subsection{Função de Aptidão}

Nesta pasta estão as implementações das funções fitness.

\textbf{FitnessHeuristic}: Para cada indivíduo da população é calculada uma aptidão com base em uma função heurística (Manhattam, Octile, Euclideana, Chebyshev)
previamente definida, o calculo é feito a partir do ponto final da lista do genoma do indivíduo até o ponto de destino do mapa
a soma de todos os resultados é a função de aptidão da população.

\textbf{FitnessWithCollisionDetection}: Para cada indivíduo da população é calculada a função de aptidão idêntica a FitnessHeuristic, porém no processo de adaptação adaptação os caminhos que aumentarem para um caminho inválido, que colidam ou saiam do mapa, são marcados,
posteriormente todos os indivíduos marcados são penalizados com a  soma de  um valor alto para diminuir o valor de sua aptidão na população.

\subsection{Adaptação}

Ela é importante para corrigir possíveis problemas nos indivíduos resultantes dos operadores de cruzamento ou mutação. Quando os cromossomos são reorganizados, 
o caminho novo que foi gerado pode levar para cima de um bloqueio ou para fora do mapa, então seguindo as direções indicadas no cromossomo, 
a posição no mapa é recalcula e só adiciona o cromossomo do indivíduo se for uma posição valida ou que não voltam para o mesmo lugar.
Para complementar o caminho do indivíduo, uma nova direção valida é adicionada e calculada, fazendo com que cada interação de adaptação, o caminho cresça.

\subsection{Mutação}

Todas as mutações executam se o indivíduo tiver mais do que 3 cromossomos e não afeta o primeiro cromossomo.
O primeiro cromossomo é a ligação com o ponto inicial e se for trocado de lugar, o caminho é quebrado. /cite{MatBuckland}

\section{Modo Batch}

O modo Batch do programa serve para poder gerar os dados necessários para a analise.
