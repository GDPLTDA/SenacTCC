\chapter[Metodologia]{Metodologia}

Nesse capitulo será apresentado mais aprofundadamente as ferramentas e métodos que foram utilizados para realizar os testes do modelo proposto.

\section{Tecnologia}

O projeto de implementação do modelo esta sendo desenvolvido em .NET, o principal Framework da Microsoft para desenvolvimento de softwares, permitindo desenvolver para multiplataformas, e oferece ferramentas satisfatórias de testes dos resultados e performance. A versão escolhida é a .NET Core RC2, que permite uso simples entre múltiplas plataformas utilizando o mesmo código fonte.

A linguagem escolhida foi C\#, a mesma possui todas as funcionalidades necessárias para implementação do nosso modelo, além de ser a que melhor implementa todas as funcionalidades do Framework .NET, e também nos da a opção futura de integração com ferramentas gráficas como por exemplo o Unity.

Para os testes o ambiente base da implementação é um Maquina Virtual do Ubuntu 64 bits rodando no Vagrant.

\section{Testes}

Foi desenvolvida uma ferramenta  para exportar mapas  manualmente a partir de um simulador de uma biblioteca de Busca de caminhos em Java Script \cite{Xueqiao}. A modificação permite exportar os o modelo desenhado em ASCII na estrutura necessária para os nossos testes, a mesma pode ser encontrada em  \hyperref[]{'http://lucasteles.github.io/pf/visual/'}.
Todos os testes seguem o mesmo modelo de mapa para exibição e leitura, para facilitar comparações.

\section{Teste de busca de caminho utilizando A*}

\section{Teste de busca de caminho utilizando Algoritmos Genéticos}
