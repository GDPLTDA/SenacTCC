\chapter[Revisão de Literatura]{Revisão de Literatura}
%\addcontentsline{toc}{chapter}{Revisão de Literatura}
% ----------------------------------------------------------
Algoritmos genéticos paralelos podem ser utilizados para a resolução do problema do caixeiro viajante, como mostra os trabalhos \cite{Jason}\cite{Alaoui}\cite{Heinz}. O Problema consiste em dado um numero finito de cidades com seus custos de viagem entre elas, deve-se encontrar o caminho mais curto para viajar entre todas as cidades e voltar ao ponto inicial. 
O trabalho de \cite{Jason}, propõem um modelo de mestre-escravo, onde o mestre fica responsável na manutenção da população e execução dos operadores genéticos. A avaliação dos melhores indivíduos é distribuída para os demais escravos, O mestre envia um indivíduo a cada um dos escravos subjacentes. Cada escravo realiza a interpretação do problema, aplica a função de cálculo para a escolha dos melhores indivíduos e envia seus resultados ao mestre, que executa seleção dos indivíduos e a geração da nova população, repetindo o processo como um todo. A estrutura foi utilizada para a automação de design de circuitos eletrônicos.
O trabalho de \cite{Alaoui}, também propõe um modelo de mestre escravo, mas cada um dos nós escravos subjacentes fica responsável por sua própria população. O nó central mestre, cria as populações iniciais e as distribui para os nós escravos. Cada nó escravo processa a evolução da população por um determinado número de gerações e então a submete ao mestre. O mestre então seleciona os melhores indivíduos dentre todas as populações dos nós escravos e os distribui novamente. Em cada nó escravo, os novos indivíduos distribuídos pelo mestre são inseridos na população corrente e o processo de evolução recomeça. O processo de migração entre os escravos, que é controlado pelo nó mestre, implementa o mecanismo que regula a velocidade da convergência e oferece os meios de escape de mínimos locais. Entretanto como relatado, a migração das populações dos nós escravos para o mestre e vice versa, pode impor um certo grau de sobre carga, dependente do meio de comunicação entre os nós. A estrutura foi utilizada para o mapeamento de tarefas em maquinas paralelas.
O trabalho de \cite{Heinz}, propõem um modelo onde cada indivíduo é o responsável por encontrar e reproduzir com um parceiro em sua vizinhança. O controle de seleção e reprodução se espalha pela população e o algoritmo deixa de ser centralizado em um mestre, com isso, diminui o grau de sincronização e facilita a paralelização. O processo do algoritmo é definir uma representação genética para o problema e criar a estrutura de vizinhança e sua população inicial. Cada indivíduo faz uma busca em sua vizinhança e seleciona um parceiro para a reprodução. Uma geração descendente é criada com o operador genético resultante.
O trabalho de \cite{Vilson}, faz uma discussão entre as estruturas \cite{Jason}\cite{Alaoui}\cite{Heinz}. No modelo de \cite{Jason} existe problema em explorar o paralelismo no calculo de verificação dos indivíduos não explorando para a reprodução e mutação. No modelo de \cite{Heinz}, tem a possibilidade de utilizar vários métodos de busca de indivíduos da mesma população, sendo úteis em casos que a eficiência dos métodos de busca se mostram dependentes da instancia do problema. O modelo \cite{Alaoui}, por todos os escravos devem enviar para o nó mestre, demanda uma grande capacidade de processamento no nó mestre, e proporciona a divisão das populações em pequenas ou de médio porte. \cite{Vilson} desenvolveu seu próprio modelo, utilizando o modelo de \cite{Alaoui} como inspiração. O modelo segue o conceito mestre-escravo, o mestre crias as populações e distribui a cada uma delas, os conjuntos de genes e parâmetros iniciais. O mestre é utilizado para a troca de indivíduos entre as populações, mantendo um indivíduo de cada população ate serem substituídos por um melhor e envia esses indivíduos para as populações que não seja a sua de origem. As populações são independentes, gerando seus indivíduos inicial com base nos genes enviados pelo mestre, aplicando seus próprios operadores de evolução e a população que determina os parceiros dos indivíduos.