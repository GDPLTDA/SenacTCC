\chapter[Revisão de Literatura]{Revisão de Literatura}

%\addcontentsline{toc}{chapter}{Revisão de Literatura}
% ----------------------------------------------------------

\section{Algoritimos geneticos paralelos para busca de caminhos}

Foi demonstrado que algoritmos genéticos paralelos são eficientes para a resolução de problemas de busca de caminho, tal como o classico problema do caixeiro viajante, que consiste em dado um numero finito de cidades com seus custos de viagem entre elas, deve-se encontrar o caminho mais curto para viajar entre todas as cidades e voltar ao ponto inicial, o problema pode ser representado pelo modelo de um grafo direcionado ponderado, aplicando a mesma ideia, o problema seria encontrar o caminho de menor custo para percorrer todos os nós, de maneira analoga, as cidades seriam os nós e a distancia entre elas o peso das arestras \cite{Jason}\cite{Alaoui}\cite{Heinz}.

*NOTA* Decorrer do POR QUE a escolha do uso de AG *NOTA*

A solução para este tipo de problema pode requer uma quantidade grande de processamento. Uma boa solução seria dividir o processamento do problema em pequenas partes e distribuir cada parte para um processador a parte, trabalhando de forma distribuida ou paralela. Varios modelos para essa finalidade foram propostos.

Um modelo interessante para parelização seria o de mestre-escravo, onde o mestre fica responsável na manutenção da população e execução dos operadores genéticos. A avaliação dos melhores indivíduos é distribuída para os demais escravos, O mestre envia um indivíduo a cada um dos escravos subjacentes. Cada escravo realiza a interpretação do problema, aplica a função de cálculo para a escolha dos melhores indivíduos e envia seus resultados ao mestre, que executa seleção dos indivíduos e a geração da nova população, repetindo o processo como um todo. Essa estrutura teve implicação satisfatoria para a automação de design de circuitos eletrônicos. \cite{Jason}

Outra forma de trabalhar com o modelo de mestre escravo, seria definir que cada um dos nós escravos subjacentes fica responsável por sua própria população. O nó central mestre, cria as populações iniciais e as distribui para os nós escravos. Cada nó escravo processa a evolução da população por um determinado número de gerações e então a submete ao mestre. O mestre então seleciona os melhores indivíduos dentre todas as populações dos nós escravos e os distribui novamente. Em cada nó escravo, os novos indivíduos distribuídos pelo mestre são inseridos na população corrente e o processo de evolução recomeça. A migração entre os escravos, que é controlado pelo nó mestre, implementa o mecanismo que regula a velocidade da convergência e oferece os meios de escape dos mínimos locais. Entretanto a migração das populações dos nós escravos para o mestre e vice versa pode impor um certo grau de sobre carga, dependente do meio de comunicação entre os nós. Esse modelo obteve sucesso no mapeamento de tarefas em maquinas paralelas. \cite{Alaoui}

Podemos partir do ponto que cada indivíduo é o responsável por encontrar e reproduzir com um parceiro em sua vizinhança. O controle de seleção e reprodução se espalha pela população e o algoritmo deixa de ser centralizado em um mestre, com isso, diminui o grau de sincronização e facilita a paralelização. O processo do algoritmo é definir uma representação genética para o problema e criar a estrutura de vizinhança e sua população inicial. Cada indivíduo faz uma busca em sua vizinhança e seleciona um parceiro para a reprodução. Uma geração descendente é criada com o operador genético resultante. \cite{Heinz}

Podemos observar alguns problemas nos modelos apresentados \cite{Vilson}, no modelo de \cite{Jason} existe problema em explorar o paralelismo no calculo de verificação dos indivíduos não explorando para a reprodução e mutação. No modelo de \cite{Heinz}, tem a possibilidade de utilizar vários métodos de busca de indivíduos da mesma população, sendo úteis em casos que a eficiência dos métodos de busca se mostram dependentes da instancia do problema. O modelo \cite{Alaoui}, por todos os escravos devem enviar para o nó mestre, demanda uma grande capacidade de processamento no nó mestre, e proporciona a divisão das populações em pequenas ou de médio porte.

 \cite{Vilson} desenvolveu seu próprio modelo, utilizando o modelo de \cite{Alaoui} como inspiração. O modelo segue o conceito mestre-escravo, o mestre crias as populações e distribui a cada uma delas, os conjuntos de genes e parâmetros iniciais. O mestre é utilizado para a troca de indivíduos entre as populações, mantendo um indivíduo de cada população ate serem substituídos por um melhor e envia esses indivíduos para as populações que não seja a sua de origem. As populações são independentes, gerando seus indivíduos inicial com base nos genes enviados pelo mestre, aplicando seus próprios operadores de evolução e a população que determina os parceiros dos indivíduos.