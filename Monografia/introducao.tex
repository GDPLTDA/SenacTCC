
\chapter[Introdução]{Introdução}
%\addcontentsline{toc}{chapter}{Introdução}
% ----------------------------------------------------------

\section{Motivação}

O problema de buscar o melhor caminho entre dois pontos tem uma grande importancia em problemas da engenhria e ciencia, tais como rotear o trafico de telefone, navegar por um labirinto ou mesmo definir o layout de trilhas impressas em uma placa eletronica.

Busca de caminhos tambem tem uma grande importancia no ambito dos jogos digitais, clara em jogos aonde um jogador compete ou coopera com uma inteligencia artificial, como jogos de tiro em primeira pessoa e de estrategia em tempo real.

O Valor de entretenimento do jogo pode ser drasticamente reduzido quando os personagens nao podem atravessar um mapa complexo de forma competente. As vezes deixando visivelmente claro para o jogador a sua incapacidade de lidar com a busca de caminho de forma satisfatoria.

Ainda é comum em jogos digitais termos mais de um agente de busca de caminho ao mesmo tempo no mesmo cenario, podendo ser muitas vezes muitos custoso computacionalmente falando. Por isso varios desenvolvedores de jogos tem juntado esforços para desenvolver soluções de busca de caminho em ambientes de recursos escassos. \cite{Pontevia}

   

% \begin{figure}
% 	\begin{minipage}{.5\textwidth}
% 		\includegraphics[scale=0.2]{Imagens/hackmageddon_motivation.png}
% 		\caption{Motivação}
% 	\end{minipage}
% \end{figure}

% \begin{description} 
	
% \end{description}


\section{Escopo}



\section{Justificativa}


\section{Objetivos}


\section{Método de trabalho}


\section{Organização do trabalho}


