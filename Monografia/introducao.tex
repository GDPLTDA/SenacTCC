
\chapter[Introdução]{Introdução}
%\addcontentsline{toc}{chapter}{Introdução}
% ----------------------------------------------------------

\section{Motivação}

O problema de buscar o melhor caminho entre dois pontos tem uma grande importância em problemas da engenharia e ciência, tais como rotear o trafico de telefone, navegar por um labirinto ou mesmo definir o layout de trilhas impressas em uma placa eletrônica.

Busca de caminhos também tem uma grande importância no âmbito dos jogos digitais, onde um jogador compete ou coopera com uma inteligencia artificial e é preciso chegar ao seu destino de forma competente, como por exemplo, jogos de tiro em primeira pessoa ou de estrategia em tempo real.

O valor de entretenimento do jogo pode ser drasticamente reduzido, quando os personagens não podem atravessar um mapa complexo de forma competente, podendo afetar a experiencia de jogo ao deixar visível para o jogador a sua incapacidade de lidar com a busca de caminho de forma satisfatória.

Ainda é comum, em jogos digitais, termos mais de um agente de busca de caminho ao mesmo tempo no mesmo cenário, podendo ser muitas vezes muito custoso computacionalmente falando. Por isso vários desenvolvedores de jogos têm juntado esforços para desenvolver soluções de busca de caminho em ambientes de recursos escassos. \cite{Pontevia}

   

% \begin{figure}
% 	\begin{minipage}{.5\textwidth}
% 		\includegraphics[scale=0.2]{Imagens/hackmageddon_motivation.png}
% 		\caption{Motivação}
% 	\end{minipage}
% \end{figure}

% \begin{description} 
	
% \end{description}


\section{Escopo}



\section{Justificativa}


\section{Objetivos}


\section{Método de trabalho}


\section{Organização do trabalho}


