
\chapter[Introdução]{Introdução}
%\addcontentsline{toc}{chapter}{Introdução}
% ----------------------------------------------------------

\section{Motivação}

O problema de buscar o melhor caminho entre dois pontos tem uma grande importância em problemas da engenharia e ciência, tais como rotear o trafico de telefone, navegar por um labirinto ou mesmo definir o layout de trilhas impressas em uma placa eletrônica.

Busca de caminhos também tem uma grande importância no âmbito dos jogos digitais, onde um jogador compete ou coopera com uma inteligencia artificial e é preciso chegar ao seu destino de forma competente, como por exemplo, jogos de tiro em primeira pessoa ou de estrategia em tempo real.

O valor de entretenimento do jogo pode ser drasticamente reduzido, quando os personagens não podem atravessar um mapa complexo de forma competente, podendo afetar a experiencia de jogo ao deixar visível para o jogador a sua incapacidade de lidar com a busca de caminho de forma satisfatória.

Ainda é comum, em jogos digitais, termos mais de um agente de busca de caminho ao mesmo tempo no mesmo cenário, podendo ser muitas vezes muito custoso computacionalmente falando. Por isso vários desenvolvedores de jogos têm juntado esforços para desenvolver soluções de busca de caminho em ambientes de recursos escassos. \cite{Pontevia}

Este cenário é a principal motivação deste trabalho que consiste em propor, implementar e mensurar resultados de uma solução para busca de melhor caminho entre dois pontos.


% \begin{figure}
% 	\begin{minipage}{.5\textwidth}
% 		\includegraphics[scale=0.2]{Imagens/hackmageddon_motivation.png}
% 		\caption{Motivação}
% 	\end{minipage}
% \end{figure}

% \begin{description} 
	
% \end{description}


\section{Justificativa}

Muitas abordagens para melhorar o desempenho ou diminuir o custo de métodos de busca de caminho tem sido desenvolvidos \cite{Ulysses}  \cite{Pollack} \cite{Pollack} \cite{Timothy} \cite{WilliamMiller}. 

Porem ainda temos problemas de alto custo computacional, que para serem minimizados em alguns casos acabamos sacrificando a certeza de melhor caminho \cite{Botea} por um melhor desempenho, por isso iremos fazer uma análise e propor um modelo para otimizar alguma limitação do modelo, visando em principal o âmbito dos jogos digitais.

\section{Objetivos}

O principal objetivo deste trabalho é propor um modelo de busca de caminho do algoritmo A* com IA em uma arquitetura paralela. O A* tem sua utilização muito popular no ramo de jogos eletrônicos e existe uma necessidade na melhoria do desempenho destes algoritmos nesse meio. \cite{Ross_Graham}

A primeira parte do modelo que iremos propor, trata-se da utilização de inteligência artificial, em especifico, algoritmos genéticos. Uma arquitetura híbrida foi demonstrada indicando sucesso na melhoria de desempenho. \cite{Ryan}

Utilizar algoritmos genéticos de forma paralela mostra uma grande melhoria de desempenho conforme o numero de threads é aumentado. \cite{Reza}

Para a utilização em jogos eletrônicos é preciso que o modelo execute em tempo real, recalculando trajeto a cada iteração gráfica, tentando a partir daí, encontrar a uma solução mais próxima da ótima. Uma forma em tempo real do A* utilizando algoritmos genéticos teve resultados positivos. \cite{Ulysses2}

O objetivo é desenvolver e implementar um modelo do A* com algoritmos genéticos em uma arquitetura paralela para execução em tempo real. 

\section{Método de trabalho}


\section{Organização do trabalho}


