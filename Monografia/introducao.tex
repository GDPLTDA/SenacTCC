
\chapter[Introdução]{Introdução}
%\addcontentsline{toc}{chapter}{Introdução}
% ----------------------------------------------------------

\section{Motivação}

O problema de buscar o melhor caminho entre dois pontos tem uma grande importância em problemas da engenharia e ciência, tais como rotear o trafico de telefone, navegar por um labirinto ou mesmo definir o layout de trilhas impressas em uma placa eletrônica.

Busca de caminhos também tem uma grande importância no âmbito dos jogos digitais, onde um jogador compete ou coopera com uma inteligencia artificial e é preciso chegar ao seu destino de forma competente, como por exemplo, jogos de tiro em primeira pessoa ou de estrategia em tempo real.

O valor de entretenimento do jogo pode ser drasticamente reduzido, quando os personagens não podem atravessar um mapa complexo de forma competente, podendo afetar a experiencia de jogo ao deixar visível para o jogador a sua incapacidade de lidar com a busca de caminho de forma satisfatória.

Ainda é comum, em jogos digitais, termos mais de um agente de busca de caminho ao mesmo tempo no mesmo cenário, podendo ser muitas vezes muito custoso computacionalmente falando. Por isso vários desenvolvedores de jogos têm juntado esforços para desenvolver soluções de busca de caminho em ambientes de recursos escassos. \cite{Pontevia}

%\section{Justificativa}

Muitas abordagens para melhorar o desempenho ou diminuir o custo de métodos de busca de caminho tem sido desenvolvidos \cite{Ulysses}  \cite{Pollack} \cite{Timothy} \cite{WilliamMiller}. 

Ainda temos problemas de alto custo computacional, que para serem minimizados em alguns casos, acabamos sacrificando a certeza de melhor caminho por um melhor desempenho\cite{Botea}. Por isso iremos fazer uma análise e propor um modelo para otimizar alguma limitação do modelo, visando em principal o âmbito dos jogos digitais.

Este cenário é a principal motivação deste trabalho que consiste em propor, implementar e mensurar resultados de uma solução para busca de melhor caminho entre dois pontos.

\section{Objetivos}

Este trabalho tem como objetivo verificar a aplicabilidade de paralelismo de AG para melhoria de precisão em busca heurística de caminhos. O ponto central é a definição de uma estrutura capaz de explorar as vantagens demonstradas no uso de AG com computação paralela e aplica-las em um modelo de busca heurística de caminho que use AG para redução de custo de memória e processamento, para isto, estes tendem a sacrificar a precisão de encontrar a menor rota. 

Implementar o modelo de busca heurística com AG e mensurar os resultados. 

Aplicar um modelo de paralelização de AG. 

Combinar o modelo paralelo com o de busca e mensurar os resultados.


\section{Método de trabalho}


\section{Organização do trabalho}


